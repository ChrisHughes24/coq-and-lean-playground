\def\paperversiondraft{draft}
\def\paperversionblind{blind}

\ifx\paperversion\paperversionblind
\else
  \def\paperversion{blind}
\fi

% !TEX TS-program = pdflatex
% !TEX encoding = UTF-8 Unicode

% This is a simple template for a LaTeX document using the "article" class.
% See "book", "report", "letter" for other types of document.

\documentclass[12pt]{article} % use larger type; default would be 10pt

\usepackage{longtable}
\usepackage{booktabs}
\usepackage{xargs}
\usepackage{xparse}
\usepackage{xifthen, xstring}
\usepackage{ulem}
\usepackage{xspace}
\usepackage{multirow}
\setlength {\marginparwidth }{2cm}
\usepackage{todonotes}
\bibliographystyle{amsalpha}
\makeatletter
\font\uwavefont=lasyb10 scaled 652
\DeclareSymbolFontAlphabet{\mathrm}    {operators}
\DeclareSymbolFontAlphabet{\mathnormal}{letters}
\DeclareSymbolFontAlphabet{\mathcal}   {symbols}
\DeclareMathAlphabet      {\mathbf}{OT1}{cmr}{bx}{n}
\DeclareMathAlphabet      {\mathsf}{OT1}{cmss}{m}{n}
\DeclareMathAlphabet      {\mathit}{OT1}{cmr}{m}{it}
\DeclareMathAlphabet      {\mathtt}{OT1}{cmtt}{m}{n}
%% \newcommand\colorwave[1][blue]{\bgroup\markoverwith{\lower3\p@\hbox{\uwavefont\textcolor{#1}{\char58}}}\ULon}
% \makeatother

% \ifx\paperversion\paperversiondraft
% \newcommand\createtodoauthor[2]{%
% \def\tmpdefault{emptystring}
% \expandafter\newcommand\csname #1\endcsname[2][\tmpdefault]{\def\tmp{##1}\ifthenelse{\equal{\tmp}{\tmpdefault}}
%    {\todo[linecolor=#2!20,backgroundcolor=#2!25,bordercolor=#2,size=\tiny]{\textbf{#1:} ##2}}
%    {\ifthenelse{\equal{##2}{}}{\colorwave[#2]{##1}\xspace}{\todo[linecolor=#2!10,backgroundcolor=#2!25,bordercolor=#2]{\tiny \textbf{#1:} ##2}\colorwave[#2]{##1}}}}}
% \else
% \newcommand\createtodoauthor[2]{%
% \expandafter\newcommand\csname #1\endcsname[2][\@nil]{}}
% \fi


%%% Examples of Article customizations
% These packages are optional, depending whether you want the features they provide.
% See the LaTeX Companion or other references for full information.

%%% PAGE DIMENSIONS
\usepackage{geometry} % to change the page dimensions
\geometry{a4paper} % or letterpaper (US) or a5paper or....
\geometry{margin=1in} % for example, change the margins to 2 inches all round
% \geometry{landscape} % set up the page for landscape
%   read geometry.pdf for detailed page layout information

\usepackage{graphicx} % support the \includegraphics command and options

% \usepackage[parfill]{parskip} % Activate to begin paragraphs with an empty line rather than an indent

\usepackage[utf8x]{inputenc}
\usepackage{amssymb}
\usepackage{listings}

\usepackage{color}
\definecolor{keywordcolor}{rgb}{0.7, 0.1, 0.1}   % red
\definecolor{commentcolor}{rgb}{0.4, 0.4, 0.4}   % grey
\definecolor{symbolcolor}{rgb}{0.0, 0.1, 0.6}    % blue
\definecolor{sortcolor}{rgb}{0.1, 0.5, 0.1}      % green
\usepackage{listings}
\def\lstlanguagefiles{lstlean.tex}
\lstset{language=lean}


%%% PACKAGES
\usepackage{inputenc}
\usepackage{booktabs} % for much better looking tables
\usepackage{array} % for better arrays (eg matrices) in maths
\usepackage{paralist} % very flexible & customisable lists (eg. enumerate/itemize, etc.)
\usepackage{verbatim} % adds environment for commenting out blocks of text & for better verbatim
\usepackage{subfig} % make it possible to include more than one captioned figure/table in a single float

\usepackage{textcomp}


% These packages are all incorporated in the memoir class to one degree or another...

%%% HEADERS & FOOTERS
\usepackage{fancyhdr} % This should be set AFTER setting up the page geometry
\pagestyle{fancy}
\renewcommand{\headrulewidth}{0pt} % customise the layout...
\lhead{\leftmark}\chead{}\rhead{\rightmark}
\lfoot{}\cfoot{\thepage}\rfoot{}

%%% SECTION TITLE APPEARANCE
\usepackage{sectsty}
\allsectionsfont{\sffamily\mdseries\upshape} % (See the fntguide.pdf for font help)
% (This matches ConTeXt defaults)

%%% ToC (table of contents) APPEARANCE
\usepackage[nottoc,notlof,notlot]{tocbibind} % Put the bibliography in the ToC
\usepackage[titles,subfigure]{tocloft} % Alter the style of the Table of Contents
\renewcommand{\cftsecfont}{\rmfamily\mdseries\upshape}
\renewcommand{\cftsecpagefont}{\rmfamily\mdseries\upshape} % No bold!

\setlength{\parindent}{0em}
\setlength{\parskip}{1em}
\usepackage{amsmath}
\usepackage{amsthm}
\usepackage{upgreek}
\usepackage{tikz-cd}
\theoremstyle{definition}
\newtheorem{thm}{Theorem}[subsection]
\theoremstyle{definition}
\newtheorem{corol}[thm]{Corollary}
\theoremstyle{definition}
\newtheorem{lemma}[thm]{Lemma}
\theoremstyle{definition}
\newtheorem{defn}[thm]{Definition}
\newtheorem{exmpl}[thm]{Example}
\usepackage{lscape}
\usepackage{hyperref}
\usepackage{titlesec}
\newcommand{\invamalg}{\mathbin{\text{\rotatebox[origin=c]{180}{$\amalg$}}}}

\setcounter{secnumdepth}{4}

\titleformat{\paragraph}
{\normalfont\normalsize}{\theparagraph}{1em}{}
\titlespacing*{\paragraph}
{0pt}{3.25ex plus 1ex minus .2ex}{1.5ex plus .2ex}

%%% END Article customizations

%%% The "real" document content comes below...

\title{Proofs Using Bicompletions}
\author{Christopher Hughes}

\begin{document}

\section{Bicompletion}

The bicompletion of a small category $\mathcal{C}$ is a category $\Lambda \mathcal{C}$ with a functor
$j : \mathcal{C} \to \Lambda \mathcal{C}$ containing all small colimits and limits, and with 
the property that for any bicomplete category $\mathcal{D}$, with a functor $F : \mathcal{C} \to \mathcal{D}$,
there is a unique bicontinuous functor $F' : \mathcal{C} \to \mathcal{D}$ such that
$j \circ F' \cong F$. 

The functor $j$ is fully faithful. Everything in this document will be proven from the fully faithfulness of
$j$ and the universal property of $\Lambda \mathcal{C}$.

There is a map from $[\mathcal{C}^{op}, Set] \to \Lambda \mathcal{C}$, and from 
$[\mathcal{C}, Set]^{op} \to \Lambda \mathcal{C}$, definable using the universal property
of the presheaf category as the (co)completion of a small category.

The bicompletion of the opposite category is the opposite of the bicompletion. i.e.
$\Lambda {\mathcal{C}^{op}} \cong (\Lambda \mathcal{C})^{op}$.

% \section{Closed structure}

% If $\mathcal{C}$ is a closed category then $\Lambda \mathcal{C}$ is also closed.
% There is a hom functor $(\mathcal{C}^{op} \times \mathcal{C})\to \mathcal{C}$.
% This can be extended to a map $\Lambda (\mathcal{C}^{op} \times \mathcal{C})\to \Lambda \mathcal{C}$.
% Since the product of bicomplete categories is also bicomplete, and the bicompletion
% of the opposite is the opposite of the bicompletion, there is a functor 
% $\Lambda (\mathcal{C}^{op} \times \mathcal{C})\to \Lambda \mathcal{C}



% \subsection{Adjunctions}
% Let $F$ be a functor $\mathcal{C} \rightarrow \mathcal{D}$, where $\mathcal{D}$ and $\mathcal{D}$
% are small categories. Then $F^{*} : \Lambda\mathcal{C} \to \Lambda\mathcal{D}$
% Then $F^{*}$ has a left adjoint. Proof. 
% We can define a functor $h : \mathcal{D} \to [C^{op}, Set]$ by setting 
% $h(d, c) := \text{hom}_{\mathcal{C}}(F(c), d)$.

% The UMP of $[D^{op}, Set]$ as the cocompletion of $\mathcal{D}$, lets us extend $h$
% to a functor $

% $F \circ i : [\mathcal{C}, Set]^{op} \to \mathcal{D}$ is continuous. Therefore, since 
% $[\mathcal{C}, Set]^{op}$ is cototal, there is a left adjoint $L$ to $F \circ i$.
% Define the left adjoint of $F$ to be $i \circ L$. 
% Then $\text{hom}((i \circ L)(X), Y)$

% \section{Totality}

% The category $\Lambda \mathcal{C}$ is total. (Not sure this is true).

% Define the left adjoint $L$ of $y : \Lambda \mathcal{C} \to [\Lambda\mathcal{C}^{op}, Set]$ to 
% take a presheaf $F$ to $i(F \circ j)$, where $i$ is the canonical map 
% $[\mathcal{C}, Set]^{op} \to \Lambda \mathcal{C}$.

% Proof of isomorphism

% $\text{hom}_{\Lambda\mathcal{C}}(i (F \circ j), X)$

\end{document}
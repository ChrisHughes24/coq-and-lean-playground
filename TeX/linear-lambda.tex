\def\paperversiondraft{draft}
\def\paperversionblind{blind}

\ifx\paperversion\paperversionblind
\else
  \def\paperversion{blind}
\fi

% !TEX TS-program = pdflatex
% !TEX encoding = UTF-8 Unicode

% This is a simple template for a LaTeX document using the "article" class.
% See "book", "report", "letter" for other types of document.

\documentclass[12pt]{article} % use larger type; default would be 10pt

\usepackage{longtable}
\usepackage{booktabs}
\usepackage{xargs}
\usepackage{xparse}
\usepackage{xifthen, xstring}
\usepackage{ulem}
\usepackage{xspace}
\usepackage{multirow}
\setlength {\marginparwidth }{2cm}
\usepackage{todonotes}
\bibliographystyle{amsalpha}
\makeatletter
\font\uwavefont=lasyb10 scaled 652
\DeclareSymbolFontAlphabet{\mathrm}    {operators}
\DeclareSymbolFontAlphabet{\mathnormal}{letters}
\DeclareSymbolFontAlphabet{\mathcal}   {symbols}
\DeclareMathAlphabet      {\mathbf}{OT1}{cmr}{bx}{n}
\DeclareMathAlphabet      {\mathsf}{OT1}{cmss}{m}{n}
\DeclareMathAlphabet      {\mathit}{OT1}{cmr}{m}{it}
\DeclareMathAlphabet      {\mathtt}{OT1}{cmtt}{m}{n}
%% \newcommand\colorwave[1][blue]{\bgroup\markoverwith{\lower3\p@\hbox{\uwavefont\textcolor{#1}{\char58}}}\ULon}
% \makeatother

% \ifx\paperversion\paperversiondraft
% \newcommand\createtodoauthor[2]{%
% \def\tmpdefault{emptystring}
% \expandafter\newcommand\csname #1\endcsname[2][\tmpdefault]{\def\tmp{##1}\ifthenelse{\equal{\tmp}{\tmpdefault}}
%    {\todo[linecolor=#2!20,backgroundcolor=#2!25,bordercolor=#2,size=\tiny]{\textbf{#1:} ##2}}
%    {\ifthenelse{\equal{##2}{}}{\colorwave[#2]{##1}\xspace}{\todo[linecolor=#2!10,backgroundcolor=#2!25,bordercolor=#2]{\tiny \textbf{#1:} ##2}\colorwave[#2]{##1}}}}}
% \else
% \newcommand\createtodoauthor[2]{%
% \expandafter\newcommand\csname #1\endcsname[2][\@nil]{}}
% \fi


%%% Examples of Article customizations
% These packages are optional, depending whether you want the features they provide.
% See the LaTeX Companion or other references for full information.

%%% PAGE DIMENSIONS
\usepackage{geometry} % to change the page dimensions
\geometry{a4paper} % or letterpaper (US) or a5paper or....
\geometry{margin=1in} % for example, change the margins to 2 inches all round
% \geometry{landscape} % set up the page for landscape
%   read geometry.pdf for detailed page layout information

\usepackage{graphicx} % support the \includegraphics command and options

% \usepackage[parfill]{parskip} % Activate to begin paragraphs with an empty line rather than an indent

\usepackage[utf8x]{inputenc}
\usepackage{amssymb}
\usepackage{listings}

\usepackage{color}
\definecolor{keywordcolor}{rgb}{0.7, 0.1, 0.1}   % red
\definecolor{commentcolor}{rgb}{0.4, 0.4, 0.4}   % grey
\definecolor{symbolcolor}{rgb}{0.0, 0.1, 0.6}    % blue
\definecolor{sortcolor}{rgb}{0.1, 0.5, 0.1}      % green
\usepackage{listings}
\def\lstlanguagefiles{lstlean.tex}
\lstset{language=lean}


%%% PACKAGES
\usepackage{inputenc}
\usepackage{booktabs} % for much better looking tables
\usepackage{array} % for better arrays (eg matrices) in maths
\usepackage{paralist} % very flexible & customisable lists (eg. enumerate/itemize, etc.)
\usepackage{verbatim} % adds environment for commenting out blocks of text & for better verbatim
\usepackage{subfig} % make it possible to include more than one captioned figure/table in a single float

\usepackage{textcomp}


% These packages are all incorporated in the memoir class to one degree or another...

%%% HEADERS & FOOTERS
\usepackage{fancyhdr} % This should be set AFTER setting up the page geometry
\pagestyle{fancy}
\renewcommand{\headrulewidth}{0pt} % customise the layout...
\lhead{\leftmark}\chead{}\rhead{\rightmark}
\lfoot{}\cfoot{\thepage}\rfoot{}

%%% SECTION TITLE APPEARANCE
\usepackage{sectsty}
\allsectionsfont{\sffamily\mdseries\upshape} % (See the fntguide.pdf for font help)
% (This matches ConTeXt defaults)

%%% ToC (table of contents) APPEARANCE
\usepackage[nottoc,notlof,notlot]{tocbibind} % Put the bibliography in the ToC
\usepackage[titles,subfigure]{tocloft} % Alter the style of the Table of Contents
\renewcommand{\cftsecfont}{\rmfamily\mdseries\upshape}
\renewcommand{\cftsecpagefont}{\rmfamily\mdseries\upshape} % No bold!

\setlength{\parindent}{0em}
\setlength{\parskip}{1em}
\usepackage{amsmath}
\usepackage{amsthm}
\usepackage{upgreek}
\usepackage{tikz-cd}
\theoremstyle{definition}
\newtheorem{thm}{Theorem}[subsection]
\theoremstyle{definition}
\newtheorem{corol}[thm]{Corollary}
\theoremstyle{definition}
\newtheorem{lemma}[thm]{Lemma}
\theoremstyle{definition}
\newtheorem{defn}[thm]{Definition}
\newtheorem{exmpl}[thm]{Example}
\usepackage{lscape}
\usepackage{hyperref}
\usepackage{titlesec}
\newcommand{\invamalg}{\mathbin{\text{\rotatebox[origin=c]{180}{$\amalg$}}}}

\setcounter{secnumdepth}{4}

\titleformat{\paragraph}
{\normalfont\normalsize}{\theparagraph}{1em}{}
\titlespacing*{\paragraph}
{0pt}{3.25ex plus 1ex minus .2ex}{1.5ex plus .2ex}

%%% END Article customizations

%%% The "real" document content comes below...

\title{Linear Lambda Notes}
\author{Christopher Hughes}

\begin{document}

\section{The problem}

A term in the linear lambda calculus can be interpreted as a morphism in any 
symmetric monoidal closed category. The category of Modules is a symmetric monoidal
closed category so we can interpret it as a morphism there. The resulting interpreted morphism will 
use the tensor products in the monoidal closed category. We would like to interpret
it without using the tensor products.

In other words, given a morphism $f$ in a free monoidal closed category ($FMCC$ below) between two objects that are 
in the image of the map $F$ from the free closed category ($FCC$ below), we want to compute a morphism in $FCC$
such that it maps to the same morphism in the category of Modules.

% https://tikzcd.yichuanshen.de/#N4Igdg9gJgpgziAXAbVABwnAlgFyxMJZABgBoAmAXVJADcBDAGwFcYkQAxAYS5AF9S6TLnyEU5CtTpNW7DgFke-QSAzY8BIhOJSGLNohDzo-KTCgBzeEVAAzAE4QAtkjIgcEJOQF3HLxACMNB5ePiAOzq7BnoE0jPQARjCMAArCGmIg9lgWABY4IDR6soYcpnxAA
\begin{tikzcd}
    &  & Mod             \\
    &  &                 \\
FCC \arrow[rruu] \arrow[rr, "F"'] &  & FMCC \arrow[uu]
\end{tikzcd}

The objects in $FCC$ are given by the following inductive type (this is not quite a free closed category).
We have a set of type constants $cT$.

\begin{lstlisting}
inductive type (cT : Type) : Type
| const : cT → type
| arrow : type → type → type
\end{lstlisting}

The morphisms in $FCC$ are given by the following inductive type. For each \lstinline{type}, there is a set of
constants of that type given by \lstinline{ct}.

\begin{lstlisting}
inductive term2 (ct : type cT → Type) : Π (A : type cT), Type
| const {T : type cT} (t : ct T) : term2 T
| app {T₁ T₂ : type cT} (f : term2 (T₁.arrow T₂)) (x : term2 T₁) : term2 T₂
| id {T₁ : type cT} : term2 (T₁.arrow T₁)
| comp {T₁ T₂ T₃ : type cT} : term2 ((T₁.arrow T₂).arrow ((T₂.arrow T₃).arrow (T₁.arrow T₃)))
| swap {T₁ T₂ T₃ : type cT} : term2 ((T₁.arrow (T₂.arrow T₃)).arrow (T₂.arrow (T₁.arrow T₃)))
\end{lstlisting}

The objects in $FMCC$ are given by the following inductive type. 
There is an obvious map \lstinline{type} to \lstinline {type3}.

\begin{lstlisting}
inductive type3 (cT : Type) : Type
| const : cT → type3
| arrow : type3 → type3 → type3
| id {} : type3
| tensor : type3 → type3 → type3
\end{lstlisting}

The morphisms in $FMCC$ are given by 

\begin{lstlisting}
inductive term3 (const_term : type cT → Type) : Π (A : type3 cT), Type
| const {T : type cT} (t : const_term T) : term3 (type3.of_type T)
| id (T : type3 cT) : term3 (T.arrow T)
| curry {T₁ T₂ T₃ : type3 cT} :
  term3 ((((T₁.tensor T₂).arrow T₃)).arrow (T₁.arrow (T₂.arrow T₃)))
| uncurry {T₁ T₂ T₃ : type3 cT} :
  term3 ((T₁.arrow (T₂.arrow T₃)).arrow ((T₁.tensor T₂).arrow T₃))
| tensor_map {T₁ T₂ T₃ T₄ : type3 cT} (f₁ : term3 (T₁.arrow T₃))
  (f₂ : term3 (T₂.arrow T₄)) : term3 ((T₁.tensor T₂).arrow (T₃.tensor T₄))
| tensor_symm {T₁ T₂ : type3 cT} : term3 ((T₁.tensor T₂).arrow (T₂.tensor T₁))
| lid₁ {T : type3 cT} : term3 ((type3.id.tensor T).arrow T)
| lid₂ {T : type3 cT} : term3 (T.arrow (type3.id.tensor T))
| app {T₁ T₂ : type3 cT} (f : term3 (T₁.arrow T₂)) (x : term3 T₁) : term3 T₂
| comp {T₁ T₂ T₃ : type3 cT} (f : term3 (T₁.arrow T₂)) :
  term3 ((T₂.arrow T₃).arrow (T₁.arrow T₃))
\end{lstlisting}

\section{Solving the problem}

We solve the problem by defining a new category $(FCC \rightarrow Set) \rightarrow Set$, 
and a functor $FMCC$ to $(FCC \rightarrow Set) \rightarrow Set$ such that the following 
commutes. By fullness of the functor $FCC$ to $(FCC \rightarrow Set) \rightarrow Set$, we can compute everything we need,
and the commutavity of the diagram will tell us we defined the right thing. 
We can define the functor $FMCC$ to $(FCC \rightarrow Set) \rightarrow Set$ by defining a closed monoidal structure on
$(FCC \rightarrow Set) \rightarrow Set$. The map from $(FCC \rightarrow Set) \rightarrow Set$ to $Mod$ and its commutativity can be proven
using the universal property of the presheaf category, and the continuity of the yoneda embedding.

% https://tikzcd.yichuanshen.de/#N4Igdg9gJgpgziAXAbVABwnAlgFyxMJZAZgBoAGAXVJADcBDAGwFcYkQBZaEAX1PUy58hFOVIBWanSat2AMQDCC3vxAZseAkQBsEqQxZtEIORyUqBG4UTLF9MoyAAUihQAIAOh4BOWAOYAFjj03t4QAO5uAMowOACUnj7+QSFhkTE4vFIwUH7wRKAAZmEAtkgAjDQ4EEhi0oZIYMyMjDSM9ABGMIwACoKaIiCMMIWZfEWlFVU1iABMNAayxnIgbZ3dfVZaxr6BY6rFEGVz07ULDo3NrUPrvf3WxsOjFiCHx5Ug1UhkN113W4NdkFVvUlq8ri83kh5p8ZsRxq9Jogfl9EOQeJQeEA
\begin{tikzcd}
    &  &  & Mod                                               &  &  &                                     \\
    &  &  &                                                   &  &  &                                     \\
    &  &  &                                                   &  &  &                                     \\
    &  &  & (FCC \rightarrow Set) \rightarrow Set \arrow[uuu] &  &  &                                     \\
    &  &  &                                                   &  &  &                                     \\
FCC \arrow[rrruuuuu] \arrow[rrrrrr, "F"'] \arrow[rrruu, "full"'] &  &  &                                                   &  &  & FMCC \arrow[llluuuuu] \arrow[llluu]
\end{tikzcd}

Universal Property of $(FCC \rightarrow Set) \rightarrow Set$. 

In the following diagram each functor is labelled $con$ if it is continuous and $cocon$ if it is cocontinuous.
The category $(FCC \rightarrow Set)^{op}$ is the completion of $FCC$, so by its universal property there is a continuous
map into $Set$. It is also true that the yoneda embedding from $FCC$ is cocontinuous. The category
$(FCC \rightarrow Set) \rightarrow Set$ is the cocompletion of $(FCC \rightarrow Set)^{op}$ so there is a 
cocontinuous map into $Mod$, and it is also true that the embedding from $(FCC \rightarrow Set)^{op}$ is continuous.




% https://tikzcd.yichuanshen.de/#N4Igdg9gJgpgziAXAbVABwnAlgFyxMJZABgBoBmAXVJADcBDAGwFcYkQAxAYS5AF9S6TLnyEUAJgrU6TVuwAU3LgAIAOqoBOWAOYALHPQ0aIAd2UBlGDgCUAPWAQ0ffoJAZseAkQAsUmgxY2RBBFHjVNHX1DYzNLG3CtPQMjUwsrFyEPUR9SYmkAuWCAWWh+aRgobXgiUAAzYwBbJDIQHAgkAEYaRnoAIxhGAAVhTzEQRP0Qf1kgkABjCAXCATrG5po2pHIVkHqIJsQu1vbEcm6+geGsr2CJnCmZQPYljN21w42TyRAe-qGR7K3SL3aZPYIvHZ7A7fTanc5-K4iG7jYEPAqzBYQyh8IA
\begin{tikzcd}
    &  &                                                                     &  & Mod                                                         \\
    &  &                                                                     &  &                                                             \\
    &  &                                                                     &  &                                                             \\
FCC \arrow[rr, "cocon"'] \arrow[rrrruuu] &  & (FCC \rightarrow Set)^{op} \arrow[rruuu, "con"'] \arrow[rr, "con"'] &  & (FCC \rightarrow Set) \rightarrow Set \arrow[uuu, "cocon"']
\end{tikzcd}

The functor $(FCC \rightarrow Set) \rightarrow Set$ to $Mod$ does not preserve all limits, however it does 
preserve all of the limits of diagrams in $FCC$, because these limits exist in the category $(FCC \rightarrow Set)^{op}$.

The category $(FCC \rightarrow Set) \rightarrow Set$ can be thought of as freely adjoining all colimits of limits to the category $FCC$.
This is necessary to define a monoidal closed structure on it because the definition of a hom object in this category is a 
colimit of a limit (although it is not actually defined like that in the code). This means that the map
$(FCC \rightarrow Set) \rightarrow Set$ will preserve hom objects and tensor objects. 

We can now show why the following commutes

% https://tikzcd.yichuanshen.de/#N4Igdg9gJgpgziAXAbVABwnAlgFyxMJZAZgBoAGAXVJADcBDAGwFcYkQBZaEAX1PUy58hFOVIBWanSat2AMQDCC3vxAZseAkQBsEqQxZtEIORyUqBG4UTLF9MoyAAUihQAIAOh4BOWAOYAFjj03t4QAO5uAMowOACUnj7+QSFhkTE4vFIwUH7wRKAAZmEAtkgAjDQ4EEhi0oZIYMyMjDSM9ABGMIwACoKaIiCMMIWZfEWlFVU1iABMNAayxnIgbZ3dfVZaxr6BY6rFEGVz07ULDo3NrUPrvf3WxsOjFiCHx5Ug1UhkN113W4NdkFVvUlq8ri83kh5p8ZsRxq9Jogfl9EOQeJQeEA
\begin{tikzcd}
    &  &  & Mod                                               &  &  &                                     \\
    &  &  &                                                   &  &  &                                     \\
    &  &  &                                                   &  &  &                                     \\
    &  &  & (FCC \rightarrow Set) \rightarrow Set \arrow[uuu] &  &  &                                     \\
    &  &  &                                                   &  &  &                                     \\
FCC \arrow[rrruuuuu] \arrow[rrrrrr, "F"'] \arrow[rrruu, "full"'] &  &  &                                                   &  &  & FMCC \arrow[llluuuuu] \arrow[llluu]
\end{tikzcd}

The large triangle containing $FCC$, $FMCC$ and $Mod$ commutes more or less by definition. The triangle on the left commutes because
the map $(FCC \rightarrow Set) \rightarrow Set$ is defined using the universal property of the free cocompletion and free completion.
The triangle on the right commutes because the functor $(FCC \rightarrow Set) \rightarrow Set$ preserves colimits of limits,
which means it preserves hom objects and tensor products, so it preserves everything in $FMCC$. (This is not that precise)
Therefore the entire diagram commutes.

\section{Bicompletions}

The idea of computing with presheaves to simplify expressions or prove equalities can be taken much further with bicompletions.
Given a category $\mathcal{C}$ it is hopefully possible to computably define a bicompletion of $\mathcal{C}$, called $\Lambda \mathcal{C}$, a 
full faithful map $i : \mathcal{C} \rightarrow \Lambda \mathcal{C}$, with the following universal property.
Given a bicomplete category $\mathcal{D}$ and a functor $F : \mathcal{C} \rightarrow \mathcal{D}$, there is a unique 
bicontinuous functor $F' : \Lambda \mathcal{C} \rightarrow \mathcal{D}$,
such that $F' \circ i = F$. (Maybe equalities of functors should have been isomorphisms there)

The construction of bicompletions is probably complicated but doable. You probably have to keep on adjoining limits and then colimits
using presheaves lots of times taking care not to adjoin limits that have already been adjoined.

The above example with monoidal closed categories could also be done using bicompletions - the bicompletion of a symmetric closed category
is symmetric monoidal closed.

\subsection{Finite Limits and Colimits}

Fullness of the functor into the bicompletion will mean that if the homsets of $\mathcal{C}$ have decidable equality then so do 
many of the homsets of $\Lambda \mathcal{C}$, I think in the closure of $\mathcal{C}$ inside $\Lambda \mathcal{C}$ 
under finite products and coproducts. If $A, B, C : \mathcal{C}$ are objects, then 
$\text{Hom}_{\Lambda \mathcal{C}} (i(A) \amalg i(B), i(C)) \cong \text{Hom}_\mathcal{C} (A, C) \times \text{Hom}_\mathcal{C} (B, C)$,
so it has decidable equality, and equality in 
$\text{Hom}_{\Lambda \mathcal{C}} (i(A) \invamalg i(B), i(C))$ should be able to be decided by adjoining
an object $X$ and maps $f : X \rightarrow A$ and $g : X \rightarrow B$ to $\mathcal{C}$ to make a new
category $\mathcal{D}$. The maps $f$ and $g$ are both epic in $\mathcal{D}$. Therefore 
$\text{Hom}_{\Lambda \mathcal{C}} (i(A) \invamalg i(B), i(C)) \cong 
\text{Hom}_{\Lambda \mathcal{D}} (i(X), i(C)) \cong \text{Hom}_{D} (X, C)$. (This is not a complete proof).

\subsection{Adjunctions}

Suppose $\mathcal{C}$ and $\mathcal{D}$ are bicomplete categories and $F$ is a forgetful functor between them with a 
left adjoint. (eg. $\mathcal{C} = Ab$, $\mathcal{D} = Grp$). Suppose $\mathcal{C}'$ and $\mathcal{D}'$ are categories
and there are functors $\mathcal{C}' \rightarrow \mathcal{C}$ and $\mathcal{D}' \rightarrow \mathcal{D}$,
and $F' : \mathcal{C}' \rightarrow \mathcal{D}'$ is a restriction of $F$. Morally $\mathcal{C}'$ and $\mathcal{D}'$ might be
small finite diagrams generated by all the homs and objects in a context for example.
Then $F'$ can be lifted to a bicontinuous map $\Lambda F' : \Lambda \mathcal{C}' \rightarrow \Lambda \mathcal{D}'$.
The functor $\Lambda{F}'$ will have a left adjoint by the adjoint functor theorem (Maybe there are size issues here).
This should commute with the adjoint of $F$ (because the functors are bicontinuous and the formula for the adjoint is 
some sort of colimit). Therefore we can eliminate this adjunctions in a similar way to how we eliminated tensor products 
in the monoidal category example.

\subsection{Conclusion}

The bicompletion should hopefully be useful to provide some tools to prove equality of morphisms using universal properties,
or to partially prove them, e.g. in the associativity of polynomials example it could hopefully reduce it to proving
associativity of natural number addition. It should be able to handle expressions containing finite limits/colimits, adjunctions,
tensor products etc. all at once.



\end{document}
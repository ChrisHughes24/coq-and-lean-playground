\def\paperversiondraft{draft}
\def\paperversionblind{blind}

\ifx\paperversion\paperversionblind
\else
  \def\paperversion{blind}
\fi

% !TEX TS-program = pdflatex
% !TEX encoding = UTF-8 Unicode

% This is a simple template for a LaTeX document using the "article" class.
% See "book", "report", "letter" for other types of document.

\documentclass[12pt]{article} % use larger type; default would be 10pt

\usepackage{longtable}
\usepackage{booktabs}
\usepackage{xargs}
\usepackage{xparse}
\usepackage{xifthen, xstring}
\usepackage{ulem}
\usepackage{xspace}
\usepackage{multirow}
\setlength {\marginparwidth }{2cm}
\usepackage{todonotes}
\bibliographystyle{amsalpha}
\makeatletter
\font\uwavefont=lasyb10 scaled 652
\DeclareSymbolFontAlphabet{\mathrm}    {operators}
\DeclareSymbolFontAlphabet{\mathnormal}{letters}
\DeclareSymbolFontAlphabet{\mathcal}   {symbols}
\DeclareMathAlphabet      {\mathbf}{OT1}{cmr}{bx}{n}
\DeclareMathAlphabet      {\mathsf}{OT1}{cmss}{m}{n}
\DeclareMathAlphabet      {\mathit}{OT1}{cmr}{m}{it}
\DeclareMathAlphabet      {\mathtt}{OT1}{cmtt}{m}{n}
%% \newcommand\colorwave[1][blue]{\bgroup\markoverwith{\lower3\p@\hbox{\uwavefont\textcolor{#1}{\char58}}}\ULon}
% \makeatother

% \ifx\paperversion\paperversiondraft
% \newcommand\createtodoauthor[2]{%
% \def\tmpdefault{emptystring}
% \expandafter\newcommand\csname #1\endcsname[2][\tmpdefault]{\def\tmp{##1}\ifthenelse{\equal{\tmp}{\tmpdefault}}
%    {\todo[linecolor=#2!20,backgroundcolor=#2!25,bordercolor=#2,size=\tiny]{\textbf{#1:} ##2}}
%    {\ifthenelse{\equal{##2}{}}{\colorwave[#2]{##1}\xspace}{\todo[linecolor=#2!10,backgroundcolor=#2!25,bordercolor=#2]{\tiny \textbf{#1:} ##2}\colorwave[#2]{##1}}}}}
% \else
% \newcommand\createtodoauthor[2]{%
% \expandafter\newcommand\csname #1\endcsname[2][\@nil]{}}
% \fi


%%% Examples of Article customizations
% These packages are optional, depending whether you want the features they provide.
% See the LaTeX Companion or other references for full information.

%%% PAGE DIMENSIONS
\usepackage{geometry} % to change the page dimensions
\geometry{a4paper} % or letterpaper (US) or a5paper or....
\geometry{margin=1in} % for example, change the margins to 2 inches all round
% \geometry{landscape} % set up the page for landscape
%   read geometry.pdf for detailed page layout information

\usepackage{graphicx} % support the \includegraphics command and options

% \usepackage[parfill]{parskip} % Activate to begin paragraphs with an empty line rather than an indent

\usepackage[utf8x]{inputenc}
\usepackage{amssymb}
\usepackage{listings}

\usepackage{color}
\definecolor{keywordcolor}{rgb}{0.7, 0.1, 0.1}   % red
\definecolor{commentcolor}{rgb}{0.4, 0.4, 0.4}   % grey
\definecolor{symbolcolor}{rgb}{0.0, 0.1, 0.6}    % blue
\definecolor{sortcolor}{rgb}{0.1, 0.5, 0.1}      % green
\usepackage{listings}
\def\lstlanguagefiles{lstlean.tex}
\lstset{language=lean}


%%% PACKAGES
\usepackage{inputenc}
\usepackage{booktabs} % for much better looking tables
\usepackage{array} % for better arrays (eg matrices) in maths
\usepackage{paralist} % very flexible & customisable lists (eg. enumerate/itemize, etc.)
\usepackage{verbatim} % adds environment for commenting out blocks of text & for better verbatim
\usepackage{subfig} % make it possible to include more than one captioned figure/table in a single float

\usepackage{textcomp}


% These packages are all incorporated in the memoir class to one degree or another...

%%% HEADERS & FOOTERS
\usepackage{fancyhdr} % This should be set AFTER setting up the page geometry
\pagestyle{fancy}
\renewcommand{\headrulewidth}{0pt} % customise the layout...
\lhead{\leftmark}\chead{}\rhead{\rightmark}
\lfoot{}\cfoot{\thepage}\rfoot{}

%%% SECTION TITLE APPEARANCE
\usepackage{sectsty}
\allsectionsfont{\sffamily\mdseries\upshape} % (See the fntguide.pdf for font help)
% (This matches ConTeXt defaults)

%%% ToC (table of contents) APPEARANCE
\usepackage[nottoc,notlof,notlot]{tocbibind} % Put the bibliography in the ToC
\usepackage[titles,subfigure]{tocloft} % Alter the style of the Table of Contents
\renewcommand{\cftsecfont}{\rmfamily\mdseries\upshape}
\renewcommand{\cftsecpagefont}{\rmfamily\mdseries\upshape} % No bold!

\setlength{\parindent}{0em}
\setlength{\parskip}{1em}
\usepackage{amsmath}
\usepackage{amsthm}
\usepackage{upgreek}
\usepackage{tikz-cd}
\theoremstyle{definition}
\newtheorem{thm}{Theorem}[subsection]
\theoremstyle{definition}
\newtheorem{corol}[thm]{Corollary}
\theoremstyle{definition}
\newtheorem{lemma}[thm]{Lemma}
\theoremstyle{definition}
\newtheorem{defn}[thm]{Definition}
\newtheorem{exmpl}[thm]{Example}
\usepackage{lscape}
\usepackage{hyperref}
\usepackage{titlesec}
\newcommand{\invamalg}{\mathbin{\text{\rotatebox[origin=c]{180}{$\amalg$}}}}

\setcounter{secnumdepth}{4}

\titleformat{\paragraph}
{\normalfont\normalsize}{\theparagraph}{1em}{}
\titlespacing*{\paragraph}
{0pt}{3.25ex plus 1ex minus .2ex}{1.5ex plus .2ex}

%%% END Article customizations

%%% The "real" document content comes below...

\title{Constructing Bicompletions}
\author{Christopher Hughes}

\begin{document}

\section{Cocompletion of a Category}
If $\mathcal{C}$ is a small category then $\mathcal{C}^{op} \rightarrow \text{Type}$ is its cocompletion.
Intuitively this is because every element of $\mathcal{C}^{op} \rightarrow \text{Type}$ is a small colimit of 
a diagram in $\mathcal{C}$. If $\mathcal{C}$ is a large category, then the cocompletion is the subcategory 
of $\mathcal{C}^{op} \rightarrow \text{Type}$ consisting of elements which are a small colimit of objects in $\mathcal{C}$.
(Check this works, does the map from this subcategory preserve colimits)? We use the notation $\Sigma {C}$ for the cocompletion
of $\mathcal{C}$ and $\Pi \mathcal{C}$ for the completion.

\section{Partial Cocompletion of a Category}

We have large categories $\mathcal{C}$ and $\mathcal{D}$ and a fully faithful cocontinuous functor $F : \mathcal{C} \to \mathcal{D}$,
such that everything in $\mathcal{D}$ is a limit of objects in $\mathcal{C}$. Then $\Sigma F$ is a cocomplete category such that
there is a cocontinuous map $i_\mathcal{C} : \mathcal{C} to \Sigma F$, and a functor 
$i_\mathcal{D} : \mathcal{D} \to \Sigma F$ such that $i_\mathcal{D} \circ F = i_\mathcal{C}$, and for any category $\mathcal{E}$
with a cocontinous map $j_\mathcal{C} : \mathcal{C} \to \mathcal{E}$ and a map $j_\mathcal{D} : \mathcal{D} \to \mathcal{E}$
such that $j_\mathcal{D} \circ F = j_\mathcal{C}$, then there is a unique cocontinuous map $\Sigma F \to \mathcal{E}$ making everything
commute.

% https://tikzcd.yichuanshen.de/#N4Igdg9gJgpgziAXAbVABwnAlgFyxMJZABgBoAmAXVJADcBDAGwFcYkQAdDgW3pwAsAxk2ABhAL4hxpdJlz5CKchWp0mrdl14DhjYABFJ02djwEiy4qoYs2iThwDKWAOa8ABADEpMkBlMKFqQALNbqdg7aQiIAokaqMFAu8ESgAGYAThDcSGQgOBBIAIw0Nhr2ngAUghA1YACUIDSM9ABGMIwACnJmiiCMMGk4TSD8MPRQ7DgA7hBjEwg0OPRYjOz8EBAA1j7pWTmIeQVIyv1tHd0B5vYDQyNlETV1uyCZ2cVLhYinLe1dPYF7BlXPxhsZXvtcp8kABmUrhdhPczgt4HEr5L4wlGQ77QxBwkDtMCTRDBACczXO-yufWBLlB9wR9i4MAAHlg4Dg4O4AITuJGEcSUcRAA
\begin{tikzcd}
    &  & \Sigma F \arrow[dddd, "\exists ! cocon"', bend left=49] \\
    &  &                                                         \\
\mathcal{C} \arrow[rr, "F(cocon)", two heads, hook] \arrow[rruu, "cocon"] \arrow[rrdd, "cocon"] &  & \mathcal{D} \arrow[uu] \arrow[dd]                       \\
    &  &                                                         \\
    &  & \mathcal{E}                                            
\end{tikzcd}

When we construct $\Sigma F$ it should be obvious that $i_\mathcal{D}$ is fully faithful and continuous. It will be constructed
as a subcategory of $PSh(\mathcal{D})$.

\subsection{Constructing the Partial Cocompletion}

We define an adjunction between $\Sigma \mathcal{D}$ and $\Pi \mathcal{D}$. It is inspired by Isbell conjugation, if $F$ is the 
identity functor, then it is Isbell conjugation.

\end{document}
\def\paperversiondraft{draft}
\def\paperversionblind{blind}

\ifx\paperversion\paperversionblind
\else
  \def\paperversion{blind}
\fi

% !TEX TS-program = pdflatex
% !TEX encoding = UTF-8 Unicode

% This is a simple template for a LaTeX document using the "article" class.
% See "book", "report", "letter" for other types of document.

\documentclass[12pt]{article} % use larger type; default would be 10pt

\usepackage{longtable}
\usepackage{booktabs}
\usepackage{xargs}
\usepackage{xparse}
\usepackage{xifthen, xstring}
\usepackage{ulem}
\usepackage{xspace}
\usepackage{multirow}
\setlength {\marginparwidth }{2cm}
\usepackage{todonotes}
\bibliographystyle{amsalpha}
\makeatletter
\font\uwavefont=lasyb10 scaled 652
\DeclareSymbolFontAlphabet{\mathrm}    {operators}
\DeclareSymbolFontAlphabet{\mathnormal}{letters}
\DeclareSymbolFontAlphabet{\mathcal}   {symbols}
\DeclareMathAlphabet      {\mathbf}{OT1}{cmr}{bx}{n}
\DeclareMathAlphabet      {\mathsf}{OT1}{cmss}{m}{n}
\DeclareMathAlphabet      {\mathit}{OT1}{cmr}{m}{it}
\DeclareMathAlphabet      {\mathtt}{OT1}{cmtt}{m}{n}
%% \newcommand\colorwave[1][blue]{\bgroup\markoverwith{\lower3\p@\hbox{\uwavefont\textcolor{#1}{\char58}}}\ULon}
% \makeatother

% \ifx\paperversion\paperversiondraft
% \newcommand\createtodoauthor[2]{%
% \def\tmpdefault{emptystring}
% \expandafter\newcommand\csname #1\endcsname[2][\tmpdefault]{\def\tmp{##1}\ifthenelse{\equal{\tmp}{\tmpdefault}}
%    {\todo[linecolor=#2!20,backgroundcolor=#2!25,bordercolor=#2,size=\tiny]{\textbf{#1:} ##2}}
%    {\ifthenelse{\equal{##2}{}}{\colorwave[#2]{##1}\xspace}{\todo[linecolor=#2!10,backgroundcolor=#2!25,bordercolor=#2]{\tiny \textbf{#1:} ##2}\colorwave[#2]{##1}}}}}
% \else
% \newcommand\createtodoauthor[2]{%
% \expandafter\newcommand\csname #1\endcsname[2][\@nil]{}}
% \fi


%%% Examples of Article customizations
% These packages are optional, depending whether you want the features they provide.
% See the LaTeX Companion or other references for full information.

%%% PAGE DIMENSIONS
\usepackage{geometry} % to change the page dimensions
\geometry{a4paper} % or letterpaper (US) or a5paper or....
\geometry{margin=1in} % for example, change the margins to 2 inches all round
% \geometry{landscape} % set up the page for landscape
%   read geometry.pdf for detailed page layout information

\usepackage{graphicx} % support the \includegraphics command and options

% \usepackage[parfill]{parskip} % Activate to begin paragraphs with an empty line rather than an indent

\usepackage[utf8x]{inputenc}
\usepackage{amssymb}
\usepackage{listings}

\usepackage{color}
\definecolor{keywordcolor}{rgb}{0.7, 0.1, 0.1}   % red
\definecolor{commentcolor}{rgb}{0.4, 0.4, 0.4}   % grey
\definecolor{symbolcolor}{rgb}{0.0, 0.1, 0.6}    % blue
\definecolor{sortcolor}{rgb}{0.1, 0.5, 0.1}      % green
\usepackage{listings}


%%% PACKAGES
\usepackage{inputenc}
\usepackage{booktabs} % for much better looking tables
\usepackage{array} % for better arrays (eg matrices) in maths
\usepackage{paralist} % very flexible & customisable lists (eg. enumerate/itemize, etc.)
\usepackage{verbatim} % adds environment for commenting out blocks of text & for better verbatim
\usepackage{subfig} % make it possible to include more than one captioned figure/table in a single float

\usepackage{textcomp}


% These packages are all incorporated in the memoir class to one degree or another...

%%% HEADERS & FOOTERS
\usepackage{fancyhdr} % This should be set AFTER setting up the page geometry
\pagestyle{fancy}
\renewcommand{\headrulewidth}{0pt} % customise the layout...
\lhead{\leftmark}\chead{}\rhead{\rightmark}
\lfoot{}\cfoot{\thepage}\rfoot{}

%%% SECTION TITLE APPEARANCE
\usepackage{sectsty}
\allsectionsfont{\sffamily\mdseries\upshape} % (See the fntguide.pdf for font help)
% (This matches ConTeXt defaults)

%%% ToC (table of contents) APPEARANCE
\usepackage[nottoc,notlof,notlot]{tocbibind} % Put the bibliography in the ToC
\usepackage[titles,subfigure]{tocloft} % Alter the style of the Table of Contents
\renewcommand{\cftsecfont}{\rmfamily\mdseries\upshape}
\renewcommand{\cftsecpagefont}{\rmfamily\mdseries\upshape} % No bold!

\setlength{\parindent}{0em}
\setlength{\parskip}{1em}
\usepackage{amsmath}
\usepackage{amsthm}
\usepackage{upgreek}
\usepackage{tikz-cd}
\theoremstyle{definition}
\newtheorem{thm}{Theorem}[subsubsection]
\theoremstyle{definition}
\newtheorem{corol}[thm]{Corollary}
\theoremstyle{definition}
\newtheorem{lemma}[thm]{Lemma}
\theoremstyle{definition}
\newtheorem{defn}[thm]{Definition}
\newtheorem{exmpl}[thm]{Example}
\usepackage{lscape}
\usepackage{hyperref}
\usepackage{titlesec}

\setcounter{secnumdepth}{4}

\titleformat{\paragraph}
{\normalfont\normalsize}{\theparagraph}{1em}{}
\titlespacing*{\paragraph}
{0pt}{3.25ex plus 1ex minus .2ex}{1.5ex plus .2ex}

%%% END Article customizations

%%% The "real" document content comes below...

\title{Parametricity Notes}
\author{Christopher Hughes}

\begin{document}

\section{Questions}

\subsection{What is the correct notion of parametric category?}

There is a notion given by \cite{Hermida2014LogicalRA} of reflexive graph category. 
I don't think this is strong enough. I would like a notion of ``parametric category'' that
includes some axioms about having enough morphisms, so that the naturality results are as strong as
possible. The morphisms are some subset of the edges and ideally we want the largest 
possible subset such that we can still make sense of composition and taking preimages of relations
along morphisms.

One answer to this is to stipulate that the relations on Types are all of the form $f(x) = g(y)$ for 
some type $Z$ functions $f : X \rightarrow Z$ and $g : Y \rightarrow Z$. This is equivalent to saying
that the relation satisfies $\forall x_1 x_2 y_1 y_2, R(x_1, y_1) \rightarrow R(x_1, y_2) 
\rightarrow R(x_2, y_1) \rightarrow R(x_2, y_2)$.
I think it is true that for any $F : Type \rightarrow Type$, if $R : X \rightarrow Y \rightarrow Prop$
satisfies the above condition then so does the corresponding relation on $F(X)$ and $F(Y)$.

Then a ``parametric category'' could be defined to be a category with binary pullback, such that 
the "edges" between objects were the pullbacks. This is better than defining an edge to be a subobject 
of the product both because it is simpler in a pure categorical language and because not all categories
generated by parametricity have products but I cannot think of any that do not have binary pullbacks.

e.g. The category $\Sigma X : Type, X \times (X \to bool)$ has pullbacks but not products. 


\subsection{How to put category structure on each type?}



In order to do Sigma types it is necessary to define something a little more general than a category.
For example a morphism of groups $\Sigma G : Type, group(G)$ should be a morphism of types, and a ``morphism''
on the corresponding groups structure. i.e. if $G, H : Type$ and $g : group(G)$ and $h : group(H)$,
then a morphism is a morphism between $G$ and $H$ and a morphism between $g$ and $h$, but $g$ and $h$ have 
different types, so it does not make sense to ask what a morphism is without a more general notion of category.

\subsubsection{Case 1 : Type -> Type}

This question is hard to answer on types where $Type$ or $Prop$ appears on the left of a Pi,
e.g. in the Type $Type\rightarrow Type$. One natural suggestion is to say that a morphism between
$F$ and $G$ has type $\Pi X Y, (X \rightarrow Y) \rightarrow (F(X) \rightarrow G(Y))$. The problem 
with this definition is that the morphisms are not composable unless $F$ and $G$ are functors and 
there is no identity on $F$ unless $F$ is a functor. The definition $\Pi X, F(X) \rightarrow G(X)$,
seems to make more sense and these two definitions are equivalent (by parametricity) if $F$ and
$G$ are functors.

Possibly it is morally wrong to attempt to define a category structure on the whole of
$Type \rightarrow Type$ and the category should just be defined on the functors. 
There are no real life examples of categories with $Type$ or $Prop$ appearing on the left of a 
Pi anyway other than functor categories.

Potentially the parametricity translation could be changed as well when computing the 
"parametric category" structure of a Type in a similar way to the way arrows are computed.
e.g. $Type \rightarrow Type$ could be translated as 
$\Pi X, F(X) \rightarrow G(X) \rightarrow Type$ when $F,G : Type \rightarrow Type$. 
In this case there would no longer be any content to the parametricity theorem, but
you wouldn't use this translation all the time.

\subsubsection{Case 2}
Another difficult case is how to put the category structure on $\Sigma X : Type, ((X \rightarrow X) \rightarrow X)$.
Use notation $\spadesuit$ for the operation $(X \rightarrow X) \rightarrow X)$.
The natural definition is a map $f : X \rightarrow Y$ such that $\forall a : X \rightarrow X, 
\forall (b : Y \rightarrow Y), f \circ a = b \circ f \rightarrow f (\spadesuit a) = \spadesuit b$.
This definition does not give composable morphisms. For this example, there is no obivous way to make relations
into objects in the category, unlike the case of groups/rings etc.

A definition that might work would be for a morphism to be a pair of a 
function $f : X \rightarrow Y$ and $f_2 : (X \rightarrow X) \rightarrow (Y 
rightarrow Y)$ such that $f_2$ as a relation is the lift of $f$.
These morphisms are probably composable but the proof is hard.

I can probably find some meaningful definition in this case, but does that mean I should and how do
I demonstrate that this is the ``correct'' definition?

\subsection{What properties does the category structure need to have?}

In order for parametricity to imply naturality, given a type $\mathcal{C}$ and $X,Y:\mathcal{C}$
the type $\text{Hom}(X,Y)$ must induce a map of relations from $[[\mathcal{C}]] Y Z$ to
$[[\mathcal{C}]] X Z$, the ``preimage''. There should be some partial order on the set of relations
$[[\mathcal{C}]] Y Z$, and the parametricity translation of 
$\lambda X Y : \mathcal{C}, \text{Hom}(X, Y)$ should be something like
$\lambda (X_1 X_2 : \mathcal{C}) (R_X : [[\mathcal{C}]] X_1 X_2) (Y_1 Y_2 : \mathcal{C})
  (R_Y : [[\mathcal{C}]]Y_1 Y_2) (f_1 : X_1 \rightarrow Y_1) (f_2 : X_2 \rightarrow Y_2), 
  X_R \le (f_1, f_2)^{-1}Y_R$.



\subsection{What are the applications of all this?}

Automatically proving naturality of things. But where does this come up, there are probably many examples
of mathematical arguments that are essentially naturality even though you would never use the word naturality
when talking about it.

Automatically prove functorality.

Arguments of the form, it suffices to prove this lemma on a free object. 
This is kind of what the ring tactic is doing.

\subsection{Find the a notion of morphism of a parametric category weaker than functor, but
  still having some structure on the map of relations}

There is a notion of a dinatural or extranatural transformation between morphisms 
$F G : \mathcal{C}^{op} \times \mathcal{C} \rightarrow \mathcal{C}$. They both give 
a notion of naturality for something $\Pi X, F(X, X) \rightarrow G(X,X)$. However not
every function $T : Type \rightarrow Type$ can be written in the $T(X) = F(X, X)$ where 
$F$ is a functor $Type^{op} \times Type \rightarrow Type$ so this is not general enough. 
However it may be true that instead of functors $Type^{op} \times Type \rightarrow Type$
you could do include a relation between the two types, using $\Sigma A B : Type, A \rightarrow B \rightarrow Prop$,
instead of $Type^{op} \times Type$. This category is completely defined in \lstinline{edge_category.lean}.
It seems that there $group : Type \rightarrow Type$ can be written as a functor
$group2 : (\Sigma A B : Type, A \rightarrow B \rightarrow Prop) \rightarrow Type$, such that 
$group G = group2(G, G, eq)$. It is not clear if this generalizes or if it is useful.

\subsection{Generalizing representable functors}
If $F : \mathcal{C} \rightarrow Type$ is a functor, then one way of defining a representation
of that functor is that it is an initial object in the category $\Sigma X : \mathcal{C}, F(X)$.
This definition has the advantage that $F$ need not be a functor. How do these generalized 
universal properties work?

Sometimes something has a universal property in one category and this means it is ``obvious'' that 
it has some universal property in another category. For example, it is ``obvious'' that the initial
object in the category of pointed groups is also the free group on one generator. Can parametricity be used 
to generate all the equivalent UMPs of an object in different categories?

\subsection {What does parametricity have to do with recursion principles for inductive types?}

The arity one translation of the type $\Sigma X : Type, X \times (X \rightarrow X)$ is 
\begin{lstlisting}
  sigma (X : Type) (X_P : X -> Type) (x : X) (h0 : X_P x) 
    (f : X -> X), forall H : X, X_P H -> X_P (f H)
\end{lstlisting}





\bibliography{references}

\end{document}
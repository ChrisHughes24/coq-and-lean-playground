\def\paperversiondraft{draft}
\def\paperversionblind{blind}

\ifx\paperversion\paperversionblind
\else
  \def\paperversion{blind}
\fi

% !TEX TS-program = pdflatex
% !TEX encoding = UTF-8 Unicode

% This is a simple template for a LaTeX document using the "article" class.
% See "book", "report", "letter" for other types of document.

\documentclass[12pt]{article} % use larger type; default would be 10pt

\usepackage{longtable}
\usepackage{booktabs}
\usepackage{xargs}
\usepackage{xparse}
\usepackage{xifthen, xstring}
\usepackage{ulem}
\usepackage{xspace}
\usepackage{multirow}
\setlength {\marginparwidth }{2cm}
\usepackage{todonotes}
\bibliographystyle{amsalpha}
\makeatletter
\font\uwavefont=lasyb10 scaled 652
\DeclareSymbolFontAlphabet{\mathrm}    {operators}
\DeclareSymbolFontAlphabet{\mathnormal}{letters}
\DeclareSymbolFontAlphabet{\mathcal}   {symbols}
\DeclareMathAlphabet      {\mathbf}{OT1}{cmr}{bx}{n}
\DeclareMathAlphabet      {\mathsf}{OT1}{cmss}{m}{n}
\DeclareMathAlphabet      {\mathit}{OT1}{cmr}{m}{it}
\DeclareMathAlphabet      {\mathtt}{OT1}{cmtt}{m}{n}
%% \newcommand\colorwave[1][blue]{\bgroup\markoverwith{\lower3\p@\hbox{\uwavefont\textcolor{#1}{\char58}}}\ULon}
% \makeatother

% \ifx\paperversion\paperversiondraft
% \newcommand\createtodoauthor[2]{%
% \def\tmpdefault{emptystring}
% \expandafter\newcommand\csname #1\endcsname[2][\tmpdefault]{\def\tmp{##1}\ifthenelse{\equal{\tmp}{\tmpdefault}}
%    {\todo[linecolor=#2!20,backgroundcolor=#2!25,bordercolor=#2,size=\tiny]{\textbf{#1:} ##2}}
%    {\ifthenelse{\equal{##2}{}}{\colorwave[#2]{##1}\xspace}{\todo[linecolor=#2!10,backgroundcolor=#2!25,bordercolor=#2]{\tiny \textbf{#1:} ##2}\colorwave[#2]{##1}}}}}
% \else
% \newcommand\createtodoauthor[2]{%
% \expandafter\newcommand\csname #1\endcsname[2][\@nil]{}}
% \fi


%%% Examples of Article customizations
% These packages are optional, depending whether you want the features they provide.
% See the LaTeX Companion or other references for full information.

%%% PAGE DIMENSIONS
\usepackage{geometry} % to change the page dimensions
\geometry{a4paper} % or letterpaper (US) or a5paper or....
\geometry{margin=1in} % for example, change the margins to 2 inches all round
% \geometry{landscape} % set up the page for landscape
%   read geometry.pdf for detailed page layout information

\usepackage{graphicx} % support the \includegraphics command and options

% \usepackage[parfill]{parskip} % Activate to begin paragraphs with an empty line rather than an indent

\usepackage[utf8x]{inputenc}
\usepackage{amssymb}
\usepackage{listings}

\usepackage{color}
\definecolor{keywordcolor}{rgb}{0.7, 0.1, 0.1}   % red
\definecolor{commentcolor}{rgb}{0.4, 0.4, 0.4}   % grey
\definecolor{symbolcolor}{rgb}{0.0, 0.1, 0.6}    % blue
\definecolor{sortcolor}{rgb}{0.1, 0.5, 0.1}      % green
\usepackage{listings}
\def\lstlanguagefiles{lstlean.tex}
\lstset{language=lean}


%%% PACKAGES
\usepackage{inputenc}
\usepackage{booktabs} % for much better looking tables
\usepackage{array} % for better arrays (eg matrices) in maths
\usepackage{paralist} % very flexible & customisable lists (eg. enumerate/itemize, etc.)
\usepackage{verbatim} % adds environment for commenting out blocks of text & for better verbatim
\usepackage{subfig} % make it possible to include more than one captioned figure/table in a single float

\usepackage{textcomp}


% These packages are all incorporated in the memoir class to one degree or another...

%%% HEADERS & FOOTERS
\usepackage{fancyhdr} % This should be set AFTER setting up the page geometry
\pagestyle{fancy}
\renewcommand{\headrulewidth}{0pt} % customise the layout...
\lhead{\leftmark}\chead{}\rhead{\rightmark}
\lfoot{}\cfoot{\thepage}\rfoot{}

%%% SECTION TITLE APPEARANCE
\usepackage{sectsty}
\allsectionsfont{\sffamily\mdseries\upshape} % (See the fntguide.pdf for font help)
% (This matches ConTeXt defaults)

%%% ToC (table of contents) APPEARANCE
\usepackage[nottoc,notlof,notlot]{tocbibind} % Put the bibliography in the ToC
\usepackage[titles,subfigure]{tocloft} % Alter the style of the Table of Contents
\renewcommand{\cftsecfont}{\rmfamily\mdseries\upshape}
\renewcommand{\cftsecpagefont}{\rmfamily\mdseries\upshape} % No bold!

\setlength{\parindent}{0em}
\setlength{\parskip}{1em}
\usepackage{amsmath}
\usepackage{amsthm}
\usepackage{upgreek}
\usepackage{tikz-cd}
\theoremstyle{definition}
\newtheorem{thm}{Theorem}[subsection]
\theoremstyle{definition}
\newtheorem{corol}[thm]{Corollary}
\theoremstyle{definition}
\newtheorem{lemma}[thm]{Lemma}
\theoremstyle{definition}
\newtheorem{defn}[thm]{Definition}
\newtheorem{exmpl}[thm]{Example}
\usepackage{lscape}
\usepackage{hyperref}
\usepackage{titlesec}
\newcommand{\invamalg}{\mathbin{\text{\rotatebox[origin=c]{180}{$\amalg$}}}}

\setcounter{secnumdepth}{4}

\titleformat{\paragraph}
{\normalfont\normalsize}{\theparagraph}{1em}{}
\titlespacing*{\paragraph}
{0pt}{3.25ex plus 1ex minus .2ex}{1.5ex plus .2ex}

%%% END Article customizations

%%% The "real" document content comes below...

\title{Polynomial Associativity}
\author{Christopher Hughes}

\begin{document}

\section{Representable Functor}

Let $\mathcal{C}$ be a category and let $F : C \to \text{Type}$
be a co-presheaf on $\mathcal{C}$. Then we will define a corepresentation of
$F$ to be the following.

\begin{enumerate}
  \item An object $X$ of $\mathcal{C}$
  \item An element $x$ of $F(X)$
  \item For every object $Y$ of $\mathcal{C}$, a map $e_Y : F(Y) \to \mathcal{C}(X, Y)$
  \item For any object $Y$ of $\mathcal{C}$, and every element $f : F(Y)$, $F(e_Y(f))(x) = f$
  \item For any object $Y$ of $\mathcal{C}$, and every morphism $f : \mathcal{C}(X, Y)$,
    $e_Y(F(f)(x)) = f$
\end{enumerate}

A representation of a functor $F : C^{op} \to \text{Type}$ is the dual concept to a corepresentable
functor. We define it explicitly here.

\begin{enumerate}
  \item An object $X$ of $\mathcal{C}$
  \item An element $x$ of $F(X^{op})$
  \item For every object $Y$ of $\mathcal{C}$, a map $r_Y : F(Y^{op}) \to \mathcal{C}(Y, X)$
  \item For any object $Y$ of $\mathcal{C}$, and every element $f : F(Y^{op})$, $F(r_Y^{op}(f))(x) = f$.
  \item For any object $Y$ of $\mathcal{C}$, and every morphism $f : \mathcal{C}(Y, X)$,
  $r_Y(F(f^{op})(x)) = f$
\end{enumerate}

We now prove that this definition is equivalent to the more usual definition. We need to show that


\begin{equation}
\mathcal{C}(X, -) \cong F
\end{equation}

Given a map $f : \mathcal{C}(X, Y)$, then $F(f)(x)$ is an element of $F(Y)$. This gives one 
direction of the isomorphism. The other direction is given by $e_Y$, the map that is
Axiom 3 of our definition of corepresentation. Axioms 4 and 5 say that these are two sided inverses
of each other. We only need to prove naturality.

To prove naturality we need to prove that given $f : \mathcal{C}(X, Y)$ and $g : \mathcal{C}(Y, Z)$,
that $F(g \circ f)(x) = F(g)(F(f)(x))$. This follows from functoriality of $F$.

\lemma {Extensionality}
If $X$ is a corepresentation of $F$ then any two maps $f, g : \mathcal{C}(X, Y)$ are equal
if $F(f)(x) = F(g)(x)$. This follows from the fact that $f \mapsto F(f)(x)$ is injective
because it is an isomorphism.

Extensionality is in fact equivalent to Axiom 5 of a corepresentation and could be used
instead of Axiom 5 as part of the definition.

\section{Adjunction}

Given a functor $F : \mathcal{C} \to \mathcal{D}$ we will define a left adjoint of $F$ to 
a corepresention of $\mathcal{D}(F(A),-)$ for every object $A$ of $\mathcal{C}$. We will call
this map of object sets $G$. We can now prove that $G$ is a functor. Explicitly,
this is the following data.

\begin{enumerate}
  \item A map of object set $G : \text{Obj}(\mathcal{D})\to \text{Obj}(\mathcal{C})$
  \item For every object $X$ of $\mathcal{D}$, a map $\eta_X : \mathcal{D}(X, F(G(X)))$
  \item For every object $X$ of $\mathcal{D}$ and $Y$ of $\mathcal{C}$, 
    a map of sets $e_{X,Y} : \mathcal{D}(X, F(Y)) \to \mathcal{C}(G(X), Y)$
  \item For every object $X$ of $\mathcal{D}$ and $Y$ of $\mathcal{C}$, 
    and every element $f : \mathcal{D}(X, F(Y))$, $F(e_{X,Y}(f)) \circ \eta_X = f$
  \item For every object $X$ of $\mathcal{D}$ and $Y$ of $\mathcal{C}$, 
  and every element $f : \mathcal{C}(G(X), Y)$, $e_{X,Y}(F(f) \circ \eta_X) = f$
\end{enumerate}


\lemma {Extensionality}
Two maps $f,g : \mathcal{C}(G(X), Y)$ are equal iff $F(f) \circ \eta_X = F(g) \circ \eta_X$.

We can prove that $G$ is in fact a functor. 
Given objects $A$ and $B$ of $\mathcal{D}$ and a map $f : \mathcal{D}(, B)$,A
then define $G(f)$ to be $e_{A,G(B)}(\eta_B \circ f)$.

Then 
\begin{equation}
  G(\text{id}_A) = e_{A,G(A)}(\eta_A \circ \text{id}_A) = 
e_{A,G(A)}(F (\text{id}_{G(A)}) \circ \eta_A) = \text{id}_{G(A)}
\end{equation}

Also for $f : \mathcal{D}(A, B)$ and $g : \mathcal{C}(B, C)$
then we apply the extensionality lemma
\begin{equation}
  \begin{aligned}
     && F(G(g \circ f)) \circ \eta_A \\
   = && F(e_{A, G(C)}(\eta_C \circ g \circ f)) \circ \eta_A \\
   = && \eta_C \circ g \circ f \\
   = && F(e_{A, G(B)}(\eta_C \circ g)) \circ \eta_B \circ f \\
   = && F(e_{A, G(B)}(\eta_C \circ g)) \circ F(e_{B, G(C)}(\eta_B \circ f)) \\
   = && F(G(g) \circ G(f))
  \end{aligned}
\end{equation}

This works similarly for right adjoints.

\section{Method for Checking Equalities}

The basic method for checking equalities of morphisms is to write every morphism in terms of 
the universal property whenever possible, and then use extensionality and then repeatedly 
rewrite using Axiom 4 of the corepresentable or representable functor axioms.

\section{Polynomial Associativity Example}

The Free Module functor which we will call $F$ is the left adjoint to the forgetful 
functor $\text{Forget} : \text{Mod}_R \to \text{Type}$.

We will call the map $A \to \text{Forget}(F(A))$, $X$ and use subscripts for application.
We might not always write the forgetful functor explicitly.

The map $(A \to \text{Forget}(B)) \to {\text{Mod}_R}(F(A), B)$ will be called $extend$.

We will define multiplication on $F(\mathbb{N})$ as a morphism of Type 
\begin{equation}
  \text{Mod}_R(F(\mathbb{N}), [F(\mathbb{N}), F(\mathbb{N})])
\end{equation} 
Square brackets indicate the hom object in $\text{Mod}_R$.

The definition of multiplication is as follows
\begin{equation}
  extend (m \mapsto extend (n \mapsto X_{m + n}))
\end{equation}

We would like to use our extensionality lemma to prove associativity of multiplication.
In order to do this, we have to state associativity as an equality of morphisms,
as opposed to an equality of elements of the free module. We use two operations
to do this, both of which are versions of linear map composition as a linear map.

For modules $A$, $B$, and $C$ we have two versions of linear map 
composition which we call $R$ and $L$.

\begin{equation}
  \begin{aligned}
    R : \text{Mod}_R([A, B], [[B,C],[A,C]]) \\
     L : \text{Mod}_R([B, C], [[A,B],[A,C]])
  \end{aligned}
\end{equation}

Then the map $a, b, c \mapsto \text{mul} (\text{mul} (a)(b))(c)$ can be written as 
\begin{equation}
  \text{Forget}(R)(\text{mul}) \circ \text{mul}
\end{equation}

Similarly the map $a, b, c \mapsto \text{mul} (a)(\text{mul} (b)(c))$ can be written as

\begin{equation}
  \text{Forget}(L) (\text{mul}) \circ (R \circ \text{mul})
\end{equation}

These linear maps both have Type $\text{Mod}_R(F(\mathbb{N}), [F(\mathbb{N}), F(\mathbb{N})])$.

We can apply the extensionality lemma three times (using functional extensionality as well).

We then have to check that for any $i, j, k : \mathbb{N}$ that

\begin{equation}
  (R)(\text{mul}) \circ \text{mul}(X_i)(X_j)(X_k) = (L) (\text{mul}) \circ (R \circ \text{mul})(X_i)(X_j)(X_k)
\end{equation}

Unfolding the definitions of linear map composition and applying Axiom 4 several times gives
the following equality.

\begin{equation}
  X_{(i + j) + k} = X_{i + (j + k)}
\end{equation}

The associativity of multiplication of polynomials was reduced to the associativity of addition
of natural numbers.

\section{Potential Improvements}

Having to unfold the definition of linear map composition is unsatisfying as well as having to 
directly apply funext. Probably it would be better to express the universal property as 
a representation of a functor $\text{Mod}_R \to \text{Mod}_R$ and to develop some
theory of representable functors in enriched categories.

Given a functor $F : \text{Mod}_R \to \text{Mod}_R$, then if $X$ is a corepresentation of $F$,
the object $[X, A]$ is a representation of the functor $B \mapsto [B, F(A)]$. The hom object 
inherits a universal property from $X$ and linear composition can probably be written in terms of this 
universal property.

% We start by defining the free module. The free module $F(\alpha)$ over a Type $\alpha$ is the module
% such that there is a natural isomorphism of functors, natural in both $M$ and $\alpha$.

% \begin{equation}
% \text{Hom}(F(\alpha), M) \cong \text{Hom}(\alpha, \text{Forget}(M))
% \end{equation}

% \subsection{Definition in Lean}

% The universal property in Lean is stated using the following four definitions or lemmas.

% \defn

% There is a map \begin{equation}X : \text{Hom}(\alpha, Forget(F(\alpha)))\end{equation}


% \defn

% There is a map \begin{equation} extend : \text{Hom}(\alpha, \text{Forget}(M)) \rightarrow \text{Hom}(F(\alpha),M) \end{equation}

% There are two properties proven 

% \lemma\label{extendX}
  
% Let $f$ be a map of sets, $\text{Hom}(\alpha, \text{Forget}(M))$. Then
%   \begin{equation}
%     Forget(extend (f)) \circ X = f 
%   \end{equation}


% \lemma{Extensionality}
  
% Let $f$ and $g$ be module homomorphisms, $\text{Hom}(F(\alpha), M)$. Then these two maps are equal whenever
%   \begin{equation}
%     Forget(f) \circ X = Forget(g) \circ X
%   \end{equation}

% The universal property follows from these two definitions and two lemmas.
% We need to prove the natural isomorphism of functors 
% $\text{Hom}(F(\alpha), M) \cong \text{Hom}(\alpha, \text{Forget}(M))$.

% The map $\text{Hom}(F(\alpha), M) \rightarrow \text{Hom}(\alpha, \text{Forget}(M))$ is given by composition
% with $X$. If $f : \text{Hom}(F(\alpha), M)$, then 
% $Forget(f) \circ X : \text{Hom}(\alpha, \text{Forget}(M))$.

% The other direction of the isomorphism is $extend$.

% To prove this is an isomorphism we need to prove two equalities.

% To prove this is an isomorphism we need to prove that for any
%  $f : \text{Hom}(\alpha, Forget(M))$, $Forget(extend (f)) \circ X = f$ and
% for any $f : \text{Hom}(F(\alpha), M)$, $extend (Forget(f) \circ X) = f$. 
% The first equality is part of our Lean definition. The second we prove by applying
% our extensionality lemma so we have to prove $Forget(extend (Forget(f) \circ X)) \circ X = Forget(f) \circ X$.
% Apply the first lemma to the right hand side proves the equality.

% We also need to prove that $F$ is a functor. Given  Types $\alpha, \beta$ and a map $f : \alpha \to \beta$,
% the map $F(f) : \text{Hom}(F(\alpha), F(\beta))$ is given by $F(f) = extend (X \circ f)$.

% To prove $F(id) = id$ we apply the extensionality lemma and we now need to prove 
% $Forget(extend (X \circ id)) \circ X = Forget(id) \circ X$. This is a direction application 
% of Lemma \ref{extendX}.

% To prove $F(g \circ f) = F(g) \circ F(f)$, then first apply extensionality and 
% we need to prove $Forget(extend (X \circ g \circ f)) \circ X = 
%   Forget(extend (X \circ g) \circ extend (X \circ f)) \circ X$. Three applications of Lemma \ref{extendX}
%   can prove this. The first three equalities in the below expression follow from Lemma \ref{extendX},
%   the final one is functoriality of the forgetful functor.
% \begin{equation}
%   \begin{aligned}
%     && Forget(extend (X \circ g \circ f)) \circ X \\
%   = && X \circ g \circ f \\
%   = && Forget(extend (X \circ g)) \circ X \circ f \\
%   = && Forget(extend (X \circ g)) \circ Forget(extend (X \circ f)) \circ X \\
%   = && Forget(extend (X \circ g) \circ extend (X \circ f)) \circ X
%   \end{aligned}
% \end{equation}

% We now prove naturality in both arguments. There are two equalities we need to prove

% \lemma 
% For any Types $\alpha, \beta$, modules $M, N$ and any map $f : \beta \to \alpha$, and any module homomorphism 
% $g : \text{Hom}(M, N)$, and any map $h : \text{Hom}(F(\alpha), M)$

% \begin{equation}
%   Forget(g \circ h \circ F(f)) \circ X = Forget(g) \circ Forget(h) \circ X \circ f
% \end{equation}

% Replacing $F(f)$ by its definition $extend (X \circ f)$ and applying Lemma \ref{extendX} will prove this.

% The other direction of naturality is given bibliography

% \lemma
% For any Types $\alpha, \beta$, modules $M, N$ and any map $f : \beta \to \alpha$, and any module homomorphism 
% $g : \text{Hom}(M, N)$, and any map $h : \text{Hom}(\alpha, Forget(M))$.

% \begin{equation}
%   extend(Forget (g) \circ h \circ f) = g \circ extend (h) \circ F(f)
% \end{equation}

% We can apply extensionality to see that it suffices to prove
% \begin{equation}
%   Forget(extend(Forget (g) \circ h \circ f)) \circ X = Forget(g \circ extend (h) \circ F(f)) \circ X
% \end{equation}

% We apply Lemma \ref{extendX} to the left hand side and replace $F(f)$ with its definition and 
% then we see it suffices to prove

% \begin{equation}
%     Forget (g) \circ h \circ f = Forget(g) \circ Forget(extend (h)) Forget(extend (X \circ f))) \circ X
% \end{equation}

% Rewriting twice with Lemma \ref{extendX} on the right hand side proves the desired result.

\end{document}
\def\paperversiondraft{draft}
\def\paperversionblind{blind}

\ifx\paperversion\paperversionblind
\else
  \def\paperversion{blind}
\fi

% !TEX TS-program = pdflatex
% !TEX encoding = UTF-8 Unicode

% This is a simple template for a LaTeX document using the "article" class.
% See "book", "report", "letter" for other types of document.

\documentclass[12pt]{article} % use larger type; default would be 10pt

\usepackage{longtable}
\usepackage{booktabs}
\usepackage{xargs}
\usepackage{xparse}
\usepackage{xifthen, xstring}
\usepackage{ulem}
\usepackage{xspace}
\usepackage{multirow}
\setlength {\marginparwidth }{2cm}
\usepackage{todonotes}
\bibliographystyle{amsalpha}
\makeatletter
\font\uwavefont=lasyb10 scaled 652
\DeclareSymbolFontAlphabet{\mathrm}    {operators}
\DeclareSymbolFontAlphabet{\mathnormal}{letters}
\DeclareSymbolFontAlphabet{\mathcal}   {symbols}
\DeclareMathAlphabet      {\mathbf}{OT1}{cmr}{bx}{n}
\DeclareMathAlphabet      {\mathsf}{OT1}{cmss}{m}{n}
\DeclareMathAlphabet      {\mathit}{OT1}{cmr}{m}{it}
\DeclareMathAlphabet      {\mathtt}{OT1}{cmtt}{m}{n}
%% \newcommand\colorwave[1][blue]{\bgroup\markoverwith{\lower3\p@\hbox{\uwavefont\textcolor{#1}{\char58}}}\ULon}
% \makeatother

% \ifx\paperversion\paperversiondraft
% \newcommand\createtodoauthor[2]{%
% \def\tmpdefault{emptystring}
% \expandafter\newcommand\csname #1\endcsname[2][\tmpdefault]{\def\tmp{##1}\ifthenelse{\equal{\tmp}{\tmpdefault}}
%    {\todo[linecolor=#2!20,backgroundcolor=#2!25,bordercolor=#2,size=\tiny]{\textbf{#1:} ##2}}
%    {\ifthenelse{\equal{##2}{}}{\colorwave[#2]{##1}\xspace}{\todo[linecolor=#2!10,backgroundcolor=#2!25,bordercolor=#2]{\tiny \textbf{#1:} ##2}\colorwave[#2]{##1}}}}}
% \else
% \newcommand\createtodoauthor[2]{%
% \expandafter\newcommand\csname #1\endcsname[2][\@nil]{}}
% \fi


%%% Examples of Article customizations
% These packages are optional, depending whether you want the features they provide.
% See the LaTeX Companion or other references for full information.

%%% PAGE DIMENSIONS
\usepackage{geometry} % to change the page dimensions
\geometry{a4paper} % or letterpaper (US) or a5paper or....
\geometry{margin=1in} % for example, change the margins to 2 inches all round
% \geometry{landscape} % set up the page for landscape
%   read geometry.pdf for detailed page layout information

\usepackage{graphicx} % support the \includegraphics command and options

% \usepackage[parfill]{parskip} % Activate to begin paragraphs with an empty line rather than an indent

\usepackage[utf8x]{inputenc}
\usepackage{amssymb}
\usepackage{listings}

\usepackage{color}
\definecolor{keywordcolor}{rgb}{0.7, 0.1, 0.1}   % red
\definecolor{commentcolor}{rgb}{0.4, 0.4, 0.4}   % grey
\definecolor{symbolcolor}{rgb}{0.0, 0.1, 0.6}    % blue
\definecolor{sortcolor}{rgb}{0.1, 0.5, 0.1}      % green
\usepackage{listings}
\def\lstlanguagefiles{lstlean.tex}
\lstset{language=lean}


%%% PACKAGES
\usepackage{inputenc}
\usepackage{booktabs} % for much better looking tables
\usepackage{array} % for better arrays (eg matrices) in maths
\usepackage{paralist} % very flexible & customisable lists (eg. enumerate/itemize, etc.)
\usepackage{verbatim} % adds environment for commenting out blocks of text & for better verbatim
\usepackage{subfig} % make it possible to include more than one captioned figure/table in a single float

\usepackage{textcomp}


% These packages are all incorporated in the memoir class to one degree or another...

%%% HEADERS & FOOTERS
\usepackage{fancyhdr} % This should be set AFTER setting up the page geometry
\pagestyle{fancy}
\renewcommand{\headrulewidth}{0pt} % customise the layout...
\lhead{\leftmark}\chead{}\rhead{\rightmark}
\lfoot{}\cfoot{\thepage}\rfoot{}

%%% SECTION TITLE APPEARANCE
\usepackage{sectsty}
\allsectionsfont{\sffamily\mdseries\upshape} % (See the fntguide.pdf for font help)
% (This matches ConTeXt defaults)

%%% ToC (table of contents) APPEARANCE
\usepackage[nottoc,notlof,notlot]{tocbibind} % Put the bibliography in the ToC
\usepackage[titles,subfigure]{tocloft} % Alter the style of the Table of Contents
\renewcommand{\cftsecfont}{\rmfamily\mdseries\upshape}
\renewcommand{\cftsecpagefont}{\rmfamily\mdseries\upshape} % No bold!

\setlength{\parindent}{0em}
\setlength{\parskip}{1em}
\usepackage{amsmath}
\usepackage{amsthm}
\usepackage{upgreek}
\usepackage{tikz-cd}
\theoremstyle{definition}
\newtheorem{thm}{Theorem}[subsection]
\theoremstyle{definition}
\newtheorem{corol}[thm]{Corollary}
\theoremstyle{definition}
\newtheorem{lemma}[thm]{Lemma}
\theoremstyle{definition}
\newtheorem{defn}[thm]{Definition}
\newtheorem{exmpl}[thm]{Example}
\usepackage{lscape}
\usepackage{hyperref}
\usepackage{titlesec}
\newcommand{\invamalg}{\mathbin{\text{\rotatebox[origin=c]{180}{$\amalg$}}}}

\setcounter{secnumdepth}{4}

\titleformat{\paragraph}
{\normalfont\normalsize}{\theparagraph}{1em}{}
\titlespacing*{\paragraph}
{0pt}{3.25ex plus 1ex minus .2ex}{1.5ex plus .2ex}

%%% END Article customizations

%%% The "real" document content comes below...

\title{Using Universal Properties to Prove Equalities of Morphisms Automatically}
\author{Christopher Hughes}

\begin{document}

\section{Introduction}

% What is the audience? Computer Science paper, but readable by mathematicians
% Use normal maths notation without mentioning Lean or Types at first
% Use one terminology for one concept. i.e. don't mix the phrases "isomorphism of functor"
  % and "natural isomorphism"
% Say something about what the problem is before giving examples,
% Relate examples to representable functors
% Mention the usual definition of representable functor
% Define extensionality properly

When formalizing mathematics, most Types have some universal property in some
category. This universal property either describes the set of homomorphisms out of
the object or the set of homomorphisms into the object. We often want to use a universal
property both to define morphisms and to check if morphisms are equal.

For example the universal property of the polynomial ring $R[X]$ over
a commutative ring $R$ is given by the following isomorphism of functors.

\begin{equation}
\text{CommRing}(R[X], -) \cong \text{CommRing}(R, -) \times \text{Forget}(-)
\end{equation}

In other words to define a ring homomorphism out of the polynomial ring $R[X]$
into a ring $S$, it suffices to provide a ring homomorphism $R \to S$ and
an element of $S$. We will call this map
$eval : \text{CommRing}(R, S) \times S \to \text{CommRing}(R[X],S)$.
$eval$ is one direction of the isomorphism.

Additionally, if we have a ring hom $f : R[X] \to S$, we can recover
a ring homomorphism $R \to S$ by composing with the canonical map
$C$ from $R \to R[X]$ and an element of $S$ by applying $f$ to $X$.
We also know how $eval$ computes on $X$ and $C$. We know that
$eval(f, s)(X) = s$ and $eval(f, s) \circ C = f$.

This natural isomorphism also gives a method to check if two
homomorphisms out of the polynomial ring are equal. Two ring homomorphisms
$f$ and $g$ from $R[X]$ into $S$ out of the polynomial ring are equal if
$f(X) = g(X)$ and $f \circ C = g \circ C$, where $C$ is the canonical
morphism $R \to R[X]$.

As an example of how to use this universal property consider the following
problem. Suppose $R$, $S$ and $T$ are commutative rings and $f : R \to S$ and
$g : S \to T$ are ring homomorphisms. Given an element $t$ of $T$ we can define
two a map $R[X] \to T$ in two different ways. Suppose $map(f)$ is the canonical
morphism $R[X] \to S[X]$ given the map $f : R \to S$.

$map(f)$ can be defined using the universal property. $map(f)$ is equal to
$eval(C \circ f, X)$

Suppose we want to check commutativity of the following diagram \newline
% https://tikzcd.yichuanshen.de/#N4Igdg9gJgpgziAXAbVABwnAlgFyxMJZABgBoAmAXVJADcBDAGwFcYkQAlZADUpAF9S6TLnyEU5CtTpNW7ACoChIDNjwEiZYtIYs2iEAGUefftJhQA5vCKgAZgCcIAWyRkQOCEkkhG9AEYwjAAKIurivjB2OCA0unIGzvRoABR2AJRK9k6uiD6eSACMNH6BIWFi7IxRMXGy+iAwuik4pJaZgtkubjQFiMUyeuxNTC1tAAQAOpMAxlgOM+MZsb4BQaFqlQYOWJYAFjFm-EA
\begin{tikzcd}
  {S[X]} \arrow[rrdd, "{eval(t,g)}"]                             &  &   \\
                                                                 &  &   \\
  {R[X]} \arrow[uu, "map(f)"] \arrow[rr, "{eval(t,g \circ f)}"'] &  & T
\end{tikzcd}

Written equational this is the same as checking the following

\begin{equation}
  eval(t, g) \circ map(f) = eval(t, g \circ f)
\end{equation}

Unfolding $map$ gives us

\begin{equation}
  eval(t, g) \circ eval(X, C \circ f) = eval(t, g \circ f)
\end{equation}

We can apply extensionality to this equation and we now have to check

\begin{equation}
  eval(t, g) (eval(X, C \circ f)(X)) = eval(t, g \circ f)(X)
\end{equation}

and

\begin{equation}
  eval(t, g) \circ eval(X, C \circ f) \circ C = eval(t, g \circ f) \circ C
\end{equation}

These equations can both be easily verified using the two equations
$\forall f\ s, eval(f, s)(X) = s$ and
$\forall f \ s,eval(f, s)(X) \circ C = f$.

% Outline of what will happen next

\section{Representable Functor}

In this section we will give a characterisation of a corepresentation
of a copresheaf $F : \mathcal{C} \to \text{Set}$ for a category $\mathcal{C}$.

Let $\mathcal{C}$ be a category and let $F : \mathcal{C} \to \text{Set}$
be a co-presheaf on $\mathcal{C}$. Then we will define a corepresentation of
$F$ to be the following. Given the following 5 things, we can recover an
object $R$ of $\mathcal{C}$ and the natural isomorphism
of functors $\mathcal{C}(R, -) \cong F$.

\begin{enumerate}
  \item An object $R$ of $\mathcal{C}$
  \item An element $u$ of $F(R)$
  \item For every object $X$ of $\mathcal{C}$, a map $e_X : F(X) \to \mathcal{C}(R, X)$
  \item For any object $X$ of $\mathcal{C}$, and every element $f : F(X)$, $F(e_X(f))(u) = f$
  \item For any two morphisms $f, g : \mathcal{C}(R, X)$, if $F(f)(u) = F(g)(u)$ then $f = g$
\end{enumerate}

A representation of a functor $F : C^{op} \to \text{Set}$ is the dual concept to a corepresentable
functor. We define it explicitly here.

\begin{enumerate}
  \item An object $R$ of $\mathcal{C}$
  \item An element $c$ of $F(R^{op})$
  \item For every object $X$ of $\mathcal{C}$, a map $r_X : F(R^{op}) \to \mathcal{C}(X, R)$
  \item For any object $X$ of $\mathcal{C}$, and every element $f \in F(X^{op})$, $F(r_X^{op}(f))(c) = f$
  \item For any two morphisms $f, g : \mathcal{C}(X, R)$, if $F(f^{op})(c) = F(g^{op})(c)$ then $f = g$
\end{enumerate}

We now prove that this definition is equivalent to the more usual definition.
The usual definition is a pair of an object $R$ of $\mathcal{C}$ and
the following isomorphism of functors.

\begin{equation}
\mathcal{C}(R, -) \cong F
\end{equation}

Given a map $f \in \mathcal{C}(R, X)$, then $F(f)(u)$ is an element of $F(X)$. This gives one
direction of the isomorphism. The other direction is given by $e_X$, the map that is
Axiom 3 of our definition of corepresentation. The fact that these two are two sided inverses
of each other is given by Axiom 4, for one direction. The other direction, i.e.
proving $e_X(F(f)(u)) = f$, is given by applying extensionality.
After applying extensionality we have to check $F(e_X(F(f)(u)))(u) = F(f)(u)$.
Axiom 4 proves this equality

To check the naturality of this isomorphism we have to check that given
any $g \in \mathcal{C}(X,Y)$ and $f \in \mathcal{C}(R, X)$
$F(g)(F(f)(u)) = F(g \circ f)(u)$. This is just functoriality of $F$.

\subsection{Examples of Representable Functors in Lean}

\subsubsection{Free Module}
We'll show the definition of the universal property in Lean as well and how each part corresponds to each
of the five things above. The free module is over a set $S$ is the corepresentation of the functor
$M \mapsto \text{Hom}_{Set}(S, \text{Forget}(M))$, from $\text{Mod} \to \text{Set}$.

1) An object $R$ of $\mathcal{C}$
\begin{lstlisting}
@[derive add_comm_group, derive module R]
def free_module (R : Type) [comm_ring R] (S : Type) : Type :=
\end{lstlisting}
The free module over a Type \lstinline{S} is an object in the category of \lstinline{R}
modules. This is a Type and a \lstinline{module R}, which in this particular implementation
are not bundled.


2) An element $u$ of $F(R)$
\begin{lstlisting}
def X (a : S) : free_module R S :=
\end{lstlisting}
This is the canonical map of sets \lstinline{S} to \lstinline{free_module R S}.

3) For every object $X$ of $\mathcal{C}$, a map $e_X : F(X) \to \mathcal{C}(R, X)$
\begin{lstlisting}
def extend (f : S → M) : free_module R S →ₗ[R] M :=
\end{lstlisting}
This is the canonical way of extending a map \lstinline{S → M} to a map
\lstinline{free_module R S →ₗ[R] M}.

4) For any object $X$ of $\mathcal{C}$, and every element $f : F(X)$, $F(e_X(f))(u) = f$
\begin{lstlisting}
@[simp] lemma extend_X (f : S → M) (a : S) : extend f (X a : free_module R S) = f a :=
\end{lstlisting}
This lemma says that when you extend a map on a basis to a map on the whole module,

5) For any two morphisms $f, g : \mathcal{C}(R, X)$, if $F(f)(u) = F(g)(u)$ then $f = g$
\begin{lstlisting}
@[ext] lemma hom_ext {f g : free_module R S →ₗ[R] M} (h : ∀ a, f (X a) = g (X a)) : f = g :=
\end{lstlisting}

\subsubsection{Product and Coproduct of Abelian Groups}

Given two abelian (additive) groups, \lstinline{A} and \lstinline {B},
the universal property of the product is given by the following.
The product of abelian groups $A$ and $B$ is the representation
of the functor $\text{Ab}(-, A) \times \text{Ab}(-, B)$, from
$\text{Ab}^{op}$ to $\text{Set}$.

1) The object is the product of sets \lstinline{A × B} with the obvious
group structure.

2) There are two projection maps \lstinline{fst : A × B →+ A} and
\lstinline{snd : A × B →+ B}. (The types \lstinline{A} and \lstinline{B}
are both explicit arguments to each of these definitions).

3) Given two maps \lstinline{C →+ A} and \lstinline{C →+ B}, we can make a map
\lstinline{C →+ A × B}
\begin{lstlisting}
def prod (f : C →+ A) (g : C →+ B) : C →+ A × B :=
\end{lstlisting}

4) We have two lemmas relating \lstinline{fst} and \lstinline{prod} and \lstinline{snd}
and \lstinline{prod}.
\begin{lstlisting}
lemma fst_comp_prod (f : C →+ A) (g : C →* B) : (fst A B).comp (f.prod g) = f :=

lemma snd_comp_prod (f : C →+ A) (g : C →* B) : (snd A B).comp (f.prod g) = g :=
\end{lstlisting}

5) Two maps into the product are equal if each component is equal
\begin{lstlisting}
lemma hom_ext {f g : C →+ A × B}
  (h1 : (fst A B).comp f = (fst A B).comp g)
  (h2 : (snd A B).comp f = (snd A B).comp g) : f = g :=
\end{lstlisting}

The coproduct of abelian groups is the same object as the product,
but with a different universal property. The coproduct of $A$ and $B$
is the corepresentation of the functor $\text{Ab}(A, -) \times \text{Ab}(B, -)$
from $\text{Ab} \to \text{Set}$.

1) The object is the product of sets \lstinline{A × B} with the obvious
group structure.

2) There are two embeddings \lstinline{inl : A →+ A × B} and
\lstinline{inr : B →+ A × B}. (The types \lstinline{A} and \lstinline{B}
are both explicit arguments to each of these definitions).

3) Given two maps \lstinline{A →+ C} and \lstinline{B →+ C}, we can make a map
\lstinline{A × B →+ C}.
\begin{lstlisting}
def coprod (f : A →+ C) (g : B →+ C) : A × B →+ C :=
\end{lstlisting}

4) We have two lemmas relating \lstinline{inl} and \lstinline{coprod} and \lstinline{inr}
and \lstinline{coprod}.
\begin{lstlisting}
lemma coprod_comp_inl (f : A →+ C) (g : B →+ C) : (f.coprod g).comp (inl A B) = f :=

lemma coprod_comp_inr (f : A →+ C) (g : B →+ C) : (f.coprod g).comp (inr A B) = g :=
\end{lstlisting}

5) We have an extensionality lemma saying two maps out of the coproduct are equal
if they are equal after composition with each embedding
\begin{lstlisting}
lemma hom_ext {f g : A × B →+ C}
  (h1 : f.comp (inl A B) = g.comp (inl A B))
  (h2 : f.comp (inr A B) = g.comp (inr A B)) : f = g :=
\end{lstlisting}

\subsubsection{Integers}
The integers are the initial ring. So they are the corepresention of the
constant functor sending every ring to the set with one element.

1) The object is the Type of integers with the normal ring structure.

2) This part is not defined in Lean. There is no real use for it, it would just be an element of
the unit type.

3) This is the canonical map from the integers into any ring.

4) This part is not defined in Lean, if it was it would just be an equality of two
elements of the unit type. This is not useful.

5) There is a lemma saying that for any ring \lstinline{R} any two maps from the integers into \lstinline{R}
are equal.

\subsubsection{Polynomials}
The polynomial ring over a commutative ring $R$ is the corepresentation of the functor
$S \mapsto S \times \text{CommRing}(R, S)$.

1) The object is the polynomial ring \lstinline{polynomial R}, with the obvious ring structure.

2) There is a ring homomorphism \lstinline{C : R →+* polynomial R}, and an element
\lstinline{X : polynomial R}

3) There is an evaluation homomorphism to evaluate a polynomial of \lstinline{R} in
an arbitrary commutative ring \lstinline{S} given an element of \lstinline{S} and
a ring homomorphism \lstinline{R →+* S}.
\begin{lstlisting}
def eval₂ (f : R →+* S) (x : S) : polynomial R →+* S :=
\end{lstlisting}

4) There are two lemmas saying what the evaluation map does to both \lstinline{X} and
\lstinline{C}.
\begin{lstlisting}
lemma eval₂_X : eval₂ f x X = x :=

lemma eval₂_comp_C : (eval₂ f x).comp C = f :=
\end{lstlisting}

5) There is a lemma stating that two ring homomorphisms out of the polynomial ring are
equal if they are equal on \lstinline{X} and equal on \lstinline{C}.
\begin{lstlisting}
lemma hom_ext {f g : polynomial R →+* S}
  (h1 : f X = g X) (h2 : f.comp C = g.comp C) : f = g :=
\end{lstlisting}

\section{Adjunction}

%Write below paragraph again. Lots of typos in this section

Given a functor $F : \mathcal{C} \to \mathcal{D}$ we will define a left adjoint of $F$ to
a corepresention of $\mathcal{D}(F(A),-)$ for every object $A$ of $\mathcal{C}$. We will call
this map of object sets $G$. We can now prove that $G$ is a functor. Explicitly,
this is the following data.

\begin{enumerate}
  \item A map of object set $G : \text{Obj}(\mathcal{D})\to \text{Obj}(\mathcal{C})$
  \item For every object $X$ of $\mathcal{D}$, a map $\eta_X \in \mathcal{D}(X, F(G(X)))$
  \item For every object $X$ of $\mathcal{D}$ and $Y$ of $\mathcal{C}$,
    a map of sets $e_{X,Y} : \mathcal{D}(X, F(Y)) \to \mathcal{C}(G(X), Y)$
  \item For every object $X$ of $\mathcal{D}$ and $Y$ of $\mathcal{C}$,
    and every element $f \in \mathcal{D}(X, F(Y))$, $F(e_{X,Y}(f)) \circ \eta_X = f$
  \item $f,g \in \mathcal{C}(G(X), Y)$ are equal iff $F(f) \circ \eta_X = F(g) \circ \eta_X$
\end{enumerate}

We can prove that $G$ is in fact a functor.
Given objects $A$ and $B$ of $\mathcal{D}$ and a map $f : \mathcal{D}(, B)$,A
then define $G(f)$ to be $e_{A,G(B)}(\eta_B \circ f)$.

Then
\begin{equation}
  G(\text{id}_A) = e_{A,G(A)}(\eta_A \circ \text{id}_A) =
e_{A,G(A)}(F (\text{id}_{G(A)}) \circ \eta_A) = \text{id}_{G(A)}
\end{equation}

Also for $f : \mathcal{D}(A, B)$ and $g : \mathcal{C}(B, C)$
then we apply the extensionality lemma
\begin{equation}
  \begin{aligned}
     && F(G(g \circ f)) \circ \eta_A \\
   = && F(e_{A, G(C)}(\eta_C \circ g \circ f)) \circ \eta_A \\
   = && \eta_C \circ g \circ f \\
   = && F(e_{A, G(B)}(\eta_C \circ g)) \circ \eta_B \circ f \\
   = && F(e_{A, G(B)}(\eta_C \circ g)) \circ F(e_{B, G(C)}(\eta_B \circ f)) \\
   = && F(G(g) \circ G(f))
  \end{aligned}
\end{equation}

This works similarly for right adjoints.

\section{Method for Checking Equalities}

The basic method for checking equalities of morphisms is to write every morphism in terms of
the universal property whenever possible, and then use extensionality as many times
as possible and then repeatedly
rewrite using Axiom 4 of the corepresentable or representable functor axioms.

\subsection{Examples of Checking Equalities}

\subsubsection{Polynomial Example}
We demonstrate how to prove that given a polynomial
\lstinline{p} over a commutative ring \lstinline{R}, commutative rings
\lstinline{S} and \lstinline{T}, ring homs,
\lstinline{f : R →+* S} and \lstinline{g : S →+* T} and \lstinline{x : T},
that
\begin{lstlisting}
eval₂ g x (map f p) = eval₂ (g.comp f) x p
\end{lstlisting}

Here, \lstinline{map f} is the canonical map \lstinline{polynomial R →+* polynomial S}
given by extending the ring homomorphism \lstinline{f}.

The first step is to write this equality as an equality of morphisms. So we should try to prove
\begin{lstlisting}
(eval₂ g x).comp (map f) = eval₂ (g.comp f) x
\end{lstlisting}

Then we write the morphism \lstinline{map} using the universal property.
\lstinline{map f} is equal to \lstinline {eval₂ (f.comp C) X}.

We now have to prove
\begin{lstlisting}
(eval₂ g x).comp (eval₂ (f.comp C) X) = eval₂ (g.comp f) x
\end{lstlisting}

We can apply the extensionality theorem for maps out of the polynomial ring.
We have two equalities to prove
\begin{lstlisting}
((eval₂ g x).comp (eval₂ (f.comp C) X)) X = eval₂ (g.comp f) x X
\end{lstlisting}
and
\begin{lstlisting}
((eval₂ g x).comp (eval₂ (f.comp C) X)).comp C = (eval₂ (g.comp f) x X).comp C
\end{lstlisting}

We will focus on the first equalities. Using the theorem about how \lstinline{eval₂}
computes on \lstinline{X} three times, we can simplify the equality to \lstinline{x = x} which is true
by reflexivity.

For the second equality we can use the theorem about how \lstinline{eval₂}
computes on \lstinline{C}. Applying it three times gives the equality
(and associativity of composition) \lstinline{g.comp f = g.comp f}, which is true by reflexivity.

\subsubsection{Polynomial Associativity Example}

% Make sure I am always explicit about Forgetful functor.

The Free Module functor which we will call $F$ is the left adjoint to the forgetful
functor $\text{Forget} : \text{Mod}_R \to \text{Set}$.

We will call the map $A \to \text{Forget}(F(A))$, $X$ and use subscripts for application.
We might not always write the forgetful functor explicitly.

The map $(A \to \text{Forget}(B)) \to {\text{Mod}_R}(F(A), B)$ will be called $extend$.

We will define multiplication on $F(\mathbb{N})$ as a morphism of Type
\begin{equation}
  \text{Mod}_R(F(\mathbb{N}), [F(\mathbb{N}), F(\mathbb{N})])
\end{equation}
Square brackets indicate the hom object in $\text{Mod}_R$.

The definition of multiplication is as follows
\begin{equation}
  extend (m \mapsto extend (n \mapsto X_{m + n}))
\end{equation}

We would like to use our extensionality lemma to prove associativity of multiplication.
In order to do this, we have to state associativity as an equality of morphisms,
as opposed to an equality of elements of the free module. We use two operations
to do this, both of which are versions of linear map composition as a linear map.

For modules $A$, $B$, and $C$ we have two versions of linear map
composition which we call $R$ and $L$.

\begin{equation}
  \begin{aligned}
    R \in \text{Mod}_R([A, B], [[B,C],[A,C]]) \\
     L \in \text{Mod}_R([B, C], [[A,B],[A,C]])
  \end{aligned}
\end{equation}

Then the map $a, b, c \mapsto \text{mul} (\text{mul} (a)(b))(c)$ can be written as
\begin{equation}
  \text{Forget}(R)(\text{mul}) \circ \text{mul}
\end{equation}

Similarly the map $a, b, c \mapsto \text{mul} (a)(\text{mul} (b)(c))$ can be written as

\begin{equation}
  \text{Forget}(L) (\text{mul}) \circ (R \circ \text{mul})
\end{equation}

These linear maps both have Type $\text{Mod}_R(F(\mathbb{N}), [F(\mathbb{N}), F(\mathbb{N})])$.

We can apply the extensionality lemma three times (using functional extensionality as well).

We then have to check that for any $i, j, k \in \mathbb{N}$ that

\begin{equation}
  (R)(\text{mul}) \circ \text{mul}(X_i)(X_j)(X_k) = (L) (\text{mul}) \circ (R \circ \text{mul})(X_i)(X_j)(X_k)
\end{equation}

Unfolding the definitions of linear map composition and applying Axiom 4 several times gives
the following equality.

\begin{equation}
  X_{(i + j) + k} = X_{i + (j + k)}
\end{equation}

The associativity of multiplication of polynomials was reduced to the associativity of addition
of natural numbers.

\section{Potential Improvements}

Having to unfold the definition of linear map composition is unsatisfying as well as having to
directly apply funext. Probably it would be better to express the universal property as
a representation of a functor $\text{Mod}_R \to \text{Mod}_R$ and to develop some
theory of representable functors in enriched categories.

Given a functor $F : \text{Mod}_R \to \text{Mod}_R$, then if $X$ is a corepresentation of $F$,
the object $[X, A]$ is a representation of the functor $B \mapsto [B, F(A)]$. The hom object
inherits a universal property from $X$ and linear composition can probably be written in terms of this
universal property.

% We start by defining the free module. The free module $F(\alpha)$ over a Type $\alpha$ is the module
% such that there is a natural isomorphism of functors, natural in both $M$ and $\alpha$.

% \begin{equation}
% \text{Hom}(F(\alpha), M) \cong \text{Hom}(\alpha, \text{Forget}(M))
% \end{equation}

% \subsection{Definition in Lean}

% The universal property in Lean is stated using the following four definitions or lemmas.

% \defn

% There is a map \begin{equation}X : \text{Hom}(\alpha, Forget(F(\alpha)))\end{equation}


% \defn

% There is a map \begin{equation} extend : \text{Hom}(\alpha, \text{Forget}(M)) \rightarrow \text{Hom}(F(\alpha),M) \end{equation}

% There are two properties proven

% \lemma\label{extendX}

% Let $f$ be a map of sets, $\text{Hom}(\alpha, \text{Forget}(M))$. Then
%   \begin{equation}
%     Forget(extend (f)) \circ X = f
%   \end{equation}


% \lemma{Extensionality}

% Let $f$ and $g$ be module homomorphisms, $\text{Hom}(F(\alpha), M)$. Then these two maps are equal whenever
%   \begin{equation}
%     Forget(f) \circ X = Forget(g) \circ X
%   \end{equation}

% The universal property follows from these two definitions and two lemmas.
% We need to prove the natural isomorphism of functors
% $\text{Hom}(F(\alpha), M) \cong \text{Hom}(\alpha, \text{Forget}(M))$.

% The map $\text{Hom}(F(\alpha), M) \rightarrow \text{Hom}(\alpha, \text{Forget}(M))$ is given by composition
% with $X$. If $f : \text{Hom}(F(\alpha), M)$, then
% $Forget(f) \circ X : \text{Hom}(\alpha, \text{Forget}(M))$.

% The other direction of the isomorphism is $extend$.

% To prove this is an isomorphism we need to prove two equalities.

% To prove this is an isomorphism we need to prove that for any
%  $f : \text{Hom}(\alpha, Forget(M))$, $Forget(extend (f)) \circ X = f$ and
% for any $f : \text{Hom}(F(\alpha), M)$, $extend (Forget(f) \circ X) = f$.
% The first equality is part of our Lean definition. The second we prove by applying
% our extensionality lemma so we have to prove $Forget(extend (Forget(f) \circ X)) \circ X = Forget(f) \circ X$.
% Apply the first lemma to the right hand side proves the equality.

% We also need to prove that $F$ is a functor. Given  Types $\alpha, \beta$ and a map $f : \alpha \to \beta$,
% the map $F(f) : \text{Hom}(F(\alpha), F(\beta))$ is given by $F(f) = extend (X \circ f)$.

% To prove $F(id) = id$ we apply the extensionality lemma and we now need to prove
% $Forget(extend (X \circ id)) \circ X = Forget(id) \circ X$. This is a direction application
% of Lemma \ref{extendX}.

% To prove $F(g \circ f) = F(g) \circ F(f)$, then first apply extensionality and
% we need to prove $Forget(extend (X \circ g \circ f)) \circ X =
%   Forget(extend (X \circ g) \circ extend (X \circ f)) \circ X$. Three applications of Lemma \ref{extendX}
%   can prove this. The first three equalities in the below expression follow from Lemma \ref{extendX},
%   the final one is functoriality of the forgetful functor.
% \begin{equation}
%   \begin{aligned}
%     && Forget(extend (X \circ g \circ f)) \circ X \\
%   = && X \circ g \circ f \\
%   = && Forget(extend (X \circ g)) \circ X \circ f \\
%   = && Forget(extend (X \circ g)) \circ Forget(extend (X \circ f)) \circ X \\
%   = && Forget(extend (X \circ g) \circ extend (X \circ f)) \circ X
%   \end{aligned}
% \end{equation}

% We now prove naturality in both arguments. There are two equalities we need to prove

% \lemma
% For any Types $\alpha, \beta$, modules $M, N$ and any map $f : \beta \to \alpha$, and any module homomorphism
% $g : \text{Hom}(M, N)$, and any map $h : \text{Hom}(F(\alpha), M)$

% \begin{equation}
%   Forget(g \circ h \circ F(f)) \circ X = Forget(g) \circ Forget(h) \circ X \circ f
% \end{equation}

% Replacing $F(f)$ by its definition $extend (X \circ f)$ and applying Lemma \ref{extendX} will prove this.

% The other direction of naturality is given bibliography

% \lemma
% For any Types $\alpha, \beta$, modules $M, N$ and any map $f : \beta \to \alpha$, and any module homomorphism
% $g : \text{Hom}(M, N)$, and any map $h : \text{Hom}(\alpha, Forget(M))$.

% \begin{equation}
%   extend(Forget (g) \circ h \circ f) = g \circ extend (h) \circ F(f)
% \end{equation}

% We can apply extensionality to see that it suffices to prove
% \begin{equation}
%   Forget(extend(Forget (g) \circ h \circ f)) \circ X = Forget(g \circ extend (h) \circ F(f)) \circ X
% \end{equation}

% We apply Lemma \ref{extendX} to the left hand side and replace $F(f)$ with its definition and
% then we see it suffices to prove

% \begin{equation}
%     Forget (g) \circ h \circ f = Forget(g) \circ Forget(extend (h)) Forget(extend (X \circ f))) \circ X
% \end{equation}

% Rewriting twice with Lemma \ref{extendX} on the right hand side proves the desired result.

\end{document}